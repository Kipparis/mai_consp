\documentclass{article}[12pt]


%Russian-specific packages
%--------------------------------------
\usepackage[T2A]{fontenc}
\usepackage[utf8]{inputenc}
\usepackage[russian]{babel}
%--------------------------------------

%Hyphenation rules
%--------------------------------------
\usepackage{hyphenat}
\hyphenation{ма-те-ма-ти-ка вос-ста-нав-ли-вать}
%--------------------------------------

\setlength{\parindent}{0em} % space before par
\setlength{\parskip}{1em}   % spacing between pars

\usepackage{enumitem} % configure lists
% \setlist{noitemsep,topsep=0pt} % separation
\setlist{nosep} % separation
\setlist[itemize,1]{label=$\triangleleft$} % triangles looking to the left

% margin from paper borders
\usepackage[margin={0.8in, 0.3in}]{geometry}

\usepackage{amsmath}
\usepackage{amssymb}
\usepackage{amsfonts}
% 1 - Name
% 2 - Goes to the index
% 3 - Definition
% internal - some text about teorem
% \newenvironment{theorem}[3]{
% beginning
%     \vspace{0.5\parindent}
%     \par\index{#2}\emph{\underline{Th.} \textbf{#1} #3.}  % teorem's name
%     \par
% }{
% ending
    % \vspace{0.2\parindent}
% }

% 1 - Name
% 2 - Goes to index
% internal - Name's definition
\newenvironment{mydef}[2]{
% beginning
    \vspace{0.5\parindent}
    \par \index{#2} \emph{\underline{Def.} \textbf{#1}} -
}{
% ending
    \vspace{0.2\parindent}
}

% TODO
% \newcommand{hidefinition}

% 1 - Name
% 2 - Given
% 3 - Task
% internal - solution
\newenvironment{example}[3]{
    \vspace{0.5\parindent}
    \par\emph{\underline{Ex.}} #1
    \par \textbf{Given:} #2
    \par \textbf{Find:} #3
    \par \textbf{Solution:}
}{
    \vspace{0.5\parindent}
}

% displaystyle в каждом мас моде
\everymath{\displaystyle}

% vertical space for mathmode
\setlength{\abovedisplayskip}{0pt}
\setlength{\belowdisplayskip}{0pt}
\setlength{\abovedisplayshortskip}{0pt}
\setlength{\belowdisplayshortskip}{0pt}

% TODO: command for drawing box around text (so I can see why some artifact are

% TODO:
% \newcommand{grad}

% indent first line after section
\usepackage{indentfirst}

\usepackage{makeidx}
\makeindex

% keep this in the very bottom
% \usepackage[pdftex]{hyperref}
% remove red boxes around links
\usepackage[pdftex,pdfborderstyle={/S/U/W 0}]{hyperref} % this disables the boxes around links


% TODO: написать неравенство минковского

\renewcommand{\d}{\,\mathrm{d}}
\newcommand{\R}{\mathbb{R}}
\newcommand{\N}{\mathbb{N}}
\newcommand{\sign}{\textmd{sign}\,}

% \usepackage[upint]{stix}    % straight int
\usepackage[integrals]{wasysym}
% \setmathfont[StylisticSet=8]{XITS Math}

\begin{document}

\tableofcontents
\pagebreak
\printindex
\pagebreak

\section{Вычисление меры множества.}
При вычислении меры мн-ва $A$ пользуемся неравенством:
\begin{displaymath}
    0 \leqslant \mu(A) \leqslant \sum\limits_{n=1}^{\infty}A_{n}
\end{displaymath}
где $A_{n}$ - элемент покрытия множества. Таким образом, чтобы решить
задачу, мы придумываем покрытие нашего множества, которое должно либо
сходится к числу (как например мера прямоугольнка в плоскости - его
площадь), либо сходится к нулю, как мера прямой в плоскости, либо
сходится к бесконечности (в задачах обычно означает что неправильно
построил покрытие)
\par \textit{Пример.} Рассмотрим меру бесконечной биссектрисы на
плоскости. Введем следующее покрытие: до единицы покрываем биссектрису
квадратами со стороной $\frac{1}{n}$, до двойки покрываем биссектрису
квадратами $\frac{1}{n^{2}}$. Таким образом, мера биссектрисы будет
ограниченна
\begin{displaymath}
    0 \leqslant \mu(A) \leqslant
    \lim\limits_{n\rightarrow \infty}\sum\limits_{i=1}^{\infty}\frac{1}{n^{i}}
\end{displaymath}
(в пределе мы уменьшаем размер квадратов) выражение под пределом представляет
собой бесконечно убывающую геометрическую прогрессию
\begin{displaymath}
    0 \leqslant \mu(A) \leqslant \lim\limits_{n\rightarrow
    \infty}\frac{1}{n-1}
\end{displaymath}
отсюда следует что
\begin{displaymath}
    \mu(A)=0
\end{displaymath}
\section{Вычисление интеграла.}
\subsection{Интеграл Лебега}
Если функция $f$ простая - разбиваем пространство на подпространства,
где подынтегральная функция принимает фиксированное значение.

\par \textit{Формальное определение}
\begin{displaymath}
    \int\limits_{A}f(x)d\mu=\sum\limits_{i=1}^{\infty}y_{i}\mu(A_{i})
\end{displaymath}
где $y_{i}$ - значение функции на выбранном интервале, $\mu(A_{i})$ -
мера выбранного интервала.

\paragraph{Пример. 1}
\begin{displaymath}
    \int\limits_{[-2;+\infty)}2^{-[x]}dx
    =\mu{[-2;-1)}\cdot2^{2} + \mu{[-1;0)}\cdot2^{1}
    + \sum\limits_{n=0}^{\infty}\mu{[n;n+1)}\cdot 2^{-n}
    =4 + 2 + \sum\limits_{n=0}^{\infty}\frac{1}{2^{n}}
    =7 + \sum\limits_{n=1}^{\infty}\frac{1}{2^{n}}
\end{displaymath}
Вычисляем сумму бесконечно убывающей геометрической прогрессии
\begin{displaymath}
    \int\limits_{[-2;+\infty)}2^{-[x]}dx
    = 7 + \frac{\frac{1}{2}}{1-\frac{1}{2}}=8
\end{displaymath}

\paragraph{Советы.} При решении задач с \textit{Канторовой лестницей} полезно знать,
что:
\begin{itemize}
    \item $A_{nk}=\left(\frac{3^{k}-2}{3^{n}};\frac{3^{k}-1}{3^{n}}\right)$
    \item $x_{0}\in A_{nk} \Rightarrow C(x_{0})=\frac{2k-1}{2^{n}}$
    \item $\mu(A_{nk})=\frac{1}{3^{n}}$
    \item $n=1..\infty,\ k=1..2^{n-1}$
\end{itemize}

\subsection{Интеграл Лебега-Стилтьеса}
Мера вычисляется по следующим правилам:
\begin{itemize}
    \item Для отрезка: $\mu_{F}[a,b)=F(b)-F(a)$
    \item Для точки: $\mu(\{c\})=F(c+0)-F(c)$
        \subitem Если в т. $c$ нет разрыва, то $\mu(\{c\})=0$
\end{itemize}
\par Интеграл формально
\begin{displaymath}
    \int\limits_{A}x(t)d\mu_{F}=\int\limits_{A}x(t)dF(t)
\end{displaymath}

\par \textit{Функция Хевисайда}\index{Функция Хевисайда} $\chi(t)$
\begin{displaymath}
    \chi(t)=\left\{\begin{array}{ll}
    0, & t \leqslant 0 \\
    1, & t > 0
    \end{array}\right.
\end{displaymath}

\paragraph{Пример. 1} Вычислить
$\int\limits_{R}x(t)d_{\chi(t-a)}=\int\limits_{R}x(t)d\chi(t-a)$
\par Решение:
\begin{eqnarray*}
&&\int\limits_{R}x(t)d\chi(t-a)\stackrel{\text{разбиваем в точке разрыва на области}}{=}
\\&&=\int\limits_{(-\infty;a)}x(t)d\chi(t-a)+\int\limits_{\{a\}}x(t)d\chi(t-a)
+\int\limits_{(a;+\infty)}x(t)d\chi(t-a)
\end{eqnarray*}
Первое слагаемое:
\begin{displaymath}
    \mu_{\chi}(-\infty,a)=\chi(a-a)-\chi(-\infty)=\chi(0)-\chi(-\infty)=0
\end{displaymath}
Третье слагаемое:
\begin{displaymath}
    \mu_{\chi}(a,+\infty)=\chi(\infty)-\chi(a-a+0)=\chi(\infty)-\chi(0+0)=1-1=0
\end{displaymath}
Остается:
\begin{eqnarray*}
    &&\int\limits_{R}x(t)d_{\chi(t-a)}=\int\limits_{\{a\}}x(t)d\chi(t-a)=x(a)\mu_{\chi}(\{a\})
    =\\&&=x(a)\left(\chi(a-a+0)-\chi(a-a)\right)=x(a)(1-0)=x(a)
\end{eqnarray*}

\subsection{Перестановка предела и интеграла}
Пользуешься теоремой Лебега:
\begin{eqnarray*}
    \lim\limits_{n\rightarrow \infty}\int\limits_{C}f\d{g}
    =\left|\begin{array}{l}
        \textmd{1) }|f|\leqslant 1\ \textmd{на}\ [0,1]\\
        \textmd{2) }\int\limits_{C}\d{g}<\infty
    \end{array} \Rightarrow\ \textmd{т. Лебега} \right|
    =\int\limits_{C}\lim\limits_{n\rightarrow \infty}f\d{g}
\end{eqnarray*}

\par С непрерывной функцией тоже можно осуществить предельный переход:
\begin{eqnarray*}
    \lim\limits_{n\rightarrow \infty}C(t^{n})=C(\lim\limits_{n\rightarrow \infty}t^{n})
    ,\ \ t\in [0,1]
\end{eqnarray*}



\paragraph{Общие советы}
\begin{enumerate}
    \item Разбей на области, с границами в точках разрыва дифференциала
    \item Ищи области в которых выражение под дифференциалом не
        изменяется - эти области можно приравнять нулю
\end{enumerate}

\paragraph{Пример 1. Кусочная функция}
\begin{eqnarray*}
    \int\limits_{[0,10]}t^{2}\d{g(t)},\ \ g(t)=\left\{\begin{array}{cc}
            t,&0\leqslant t\leqslant 1\\
            t+3,&1< t\leqslant 4\\
            7,&4<t\leqslant 6\\
            t^{2},&t>6
    \end{array} \right.
\end{eqnarray*}
\begin{eqnarray*}
    &&\int\limits_{[0,10]}t^{2}\d{g(t)}=\int\limits_{[0,1)}t^{2}\d{t}
    +\int\limits_{(1,4]}t^{2}\d{(t+3)}+\int\limits_{(4,6)}t^{2}\d{7}
    +\int\limits_{(6,10)}t^{2}\d{t^{2}}
    +\int\limits_{\left\{1\right\}}t^{2}\d{g}
    +\int\limits_{\left\{6\right\}}t^{2}\d{g}=\\
    &&=\int\limits_{0}^{1}t^{2}\d{t}+\int\limits_{1}^{4}t^{2}\d{t}
    +\int\limits_{6}^{10}2t^{3}\d{t}
    +1^{2}\mu_{g}\left\{1\right\}+6^{2}\mu_{g}\left\{6\right\}=\\
    &&=\left.\frac{t^{3}}{3}\right|^{4}_{0}+\left.\frac{t^{4}}{2}\right|^{10}_{6}
    +1(g(1+0)-g(1))+36(g(6+0)-g(6))=\\
    &&=\frac{64}{3}+\frac{10^{4}-6^{4}}{2}+1(4-1)+36(36-7)
\end{eqnarray*}
\textit{точки разрыва под дифференциалом прорабатываем отдельно как интеграл Лебега-Стилтьеса}

\section{Сжимающие отображения. Доказательство существования
неподвижной точки.}
\paragraph{Пример 1.}
Алгоритм кратко:
\begin{enumerate}
    \item Определили рекурентную формулу для последовательности
    \item Определили в каких границах лежит $x_{n}$
    \item Определили в каких границах лежит отображение (рекурентное соотношение)
        $f(x)=2+\frac{1}{x_{n}}$
    \item Т.к. отображение лежит внутри исходного отрезка, это отображение сжимающее
    \item Выбираем метрику, считаем $\rho(f(x),f(y))$ (просто подставляем значения
        и пытаемся вывести оттуда ту же метрику, только относительно исходного отрезка)
    \item Подсчитали неподвижную точку предельным переходом
        $x^{*}=\lim\limits_{n\rightarrow \infty}x_{n}$
        (подставили в рекурентное соотношение и нашли значение)
\end{enumerate}

Есть числовая последовательность
$2;\ \ 2+\frac{1}{2};\ \ 2+\frac{1}{2+\frac{1}{2}};\ \ \ldots$. Надо определить
является ли последовательность сходящейся. Если является - найти её предел.
\par \textit{Решение:}
\begin{eqnarray*}
    &&x_{n+1}=2+\frac{1}{x_{n}},\ \ x_{1}=2,\ \ n=1,2,\ldots\\
    &&x_{n}\geqslant 2 \Rightarrow \frac{1}{x_{n}} \leqslant \frac{1}{2}\\
    &&2\leqslant x_{n+1}\leqslant 2+\frac{1}{2}=\frac{5}{2} \Rightarrow x_{n}\in
    \left[2;\frac{5}{2}\right]
\end{eqnarray*}
\par Рассммотрим отображение $f(x)=2+\frac{1}{x}$ на отрезке $x\in [2;\frac{5}{2}] \Rightarrow $
\begin{eqnarray*}
&\frac{2}{5}\leqslant \frac{1}{x} \leqslant \frac{1}{2}
\Rightarrow 2+\frac{2}{5}\leqslant 2+\frac{1}{x}\leqslant 2+\frac{1}{2}\Rightarrow\\
&\Rightarrow \frac{12}{5}\leqslant f(x)\leqslant \frac{5}{2}
\end{eqnarray*}
Полученный отрезок находится внутри исходного отрезка:
\begin{eqnarray*}
    \left[\frac{12}{5},\frac{5}{2}\right]\subset\left[2;\frac{5}{2}\right]
    \Rightarrow f(x)\textmd{ отображает } \left[2;\frac{5}{2}\right]\textmd{ в себя.}
\end{eqnarray*}
Можно проверить является ли отображение сжимающим:
\begin{eqnarray*}
    \rho(f(x),f(y))\leqslant \alpha\cdot\rho(x,y),\ (\alpha\alpha<1)
\end{eqnarray*}
\par Нужно определить метрику: $\R$, $\rho(a,b)=|a-b|$. Таким образом
\begin{eqnarray*}
    &\rho(f(x),f(y))&=\big|f(x)-f(y)\big|=\Bigg|2+\frac{1}{x}-\Big(2+\frac{1}{y}\Big)\Bigg|=\\
    &&=\Big|\frac{1}{x}-\frac{1}{y}\Big|=\Big|\frac{y-x}{xy}\Big|
    =\frac{|x-y|}{|x|\cdot|y|}=\\
    &&=\frac{1}{|x|\cdot|y|}\cdot\rho(x,y)\leqslant
\end{eqnarray*}
Вспоминаем в каких пределах у нас $\frac{1}{x} \Big(\frac{1}{x}\leqslant \frac{1}{2}\Big)$
\begin{eqnarray*}
    \leqslant \frac{1}{4}\cdot \rho(x,y)
\end{eqnarray*}
$\Rightarrow $ отображение сжимающее (пр-во полное) $\Rightarrow $ отображение
имеет единственную неподвижную точку:
\begin{eqnarray*}
    x^{*}=\lim\limits_{n\rightarrow \infty}x_{n},\ \textmd{где}\ x_{n}=f(x_{n-1})
\end{eqnarray*}
вспоминаем что
\begin{eqnarray*}
    x_{n+1}=2+\frac{1}{x_{n}}
\end{eqnarray*}
Переходим к пределу
\begin{eqnarray*}
    &&x^{*}=2+\frac{1}{x^{*}}\\
    &&x^{*}=1\pm\sqrt{2}\\
    &&2\leqslant 1+\sqrt{2}\leqslant \frac{5}{2}
\end{eqnarray*}
$\Rightarrow $ посл-ть сходится к точке $x^{*}=1+\sqrt{2}$

\paragraph{Пример 2.}
Отображение задано следующим образом:
\begin{eqnarray*}
    f:\ C[0,1]\ \to\ C[0,1]\\
    f(x)=\frac{5}{6}t+\frac{1}{2}\int\limits_{0}^{1}t\tau x(\tau)\d{\tau}
\end{eqnarray*}
\par Определяем метрику: пространство $C[0,1]$, метрика
$\rho(x,y)=\max_{0\leqslant t\leqslant 1}{|x(t)-y(t)|}$
\par Подставляем:
\begin{eqnarray*}
    \rho(f(x),f(y))=\max_{0\leqslant t\leqslant 1}|f(x)-f(y)|
    =\frac{1}{2}\max_{0\leqslant t\leqslant 1}\Big|\int\limits_{0}^{1}
    t\tau(x(\tau)-y(\tau))\d{\tau}\Big|\leqslant
\end{eqnarray*}
\par Оценим модуль
\begin{eqnarray*}
    \leqslant \frac{1}{2}\max_{0\leqslant t\leqslant 1}\int\limits_{0}^{1}
    |t|\cdot|\tau|\cdot|x(\tau)-y(\tau)|\d{\tau}
\end{eqnarray*}
\begin{itemize}
    \item $|t|$ не зависит от подинтегрального выражение $\Rightarrow $ можем вынести
    \item $|x(\tau)-y(\tau)|\leqslant \max_{0\leqslant\tau\leqslant 1}|x(\tau)-y(\tau)|=\rho(x,y)$ ограничено
\end{itemize}
Продолжаем интеграл
\begin{eqnarray*}
    \leqslant \frac{1}{2}\rho(x,y)\cdot\max_{0\leqslant t\leqslant 1}|t|
    \cdot\int\limits_{0}^{1}|\tau|\d{\tau}
    \leqslant \frac{1}{4}\rho(x,y)
\end{eqnarray*}
$\Rightarrow $ отображение явл. сжимающим $\Rightarrow $ существует неподвижная точка $x^{*}=\lim\limits_{n\rightarrow \infty}x_{n}$.
\par Надо для начала построить последовательность. Возьмем в качестве начального элемента - ноль:
\begin{eqnarray*}
    &&x_{0}(t)=0\\
    &&x_{1}=f(x_{0})=\frac{5}{6}t+\frac{1}{2}\int\limits_{0}^{1}t\tau\cdot
    x_{0}(\tau)\d{\tau}=\frac{5}{6}t\\
    &&x_{2}(t)=f(x_{1})=\frac{5}{6}t+\frac{1}{2}\int\limits_{0}^{1}t\tau
    x_{1}(\tau)\d{\tau}=\frac{5}{6}t+\frac{1}{2}\int\limits_{0}^{1}t\tau
    \frac{5}{6}\tau\d{\tau}
    =\frac{5}{6}t+\frac{1}{6}\cdot\frac{5}{6}t
    =\Big(\frac{5}{6}+\frac{5}{6^{2}}\Big)t\\
    &&x_{3}(t)=\Big(\frac{5}{6}+\frac{5}{6^{2}}+\frac{5}{6^{3}}\Big)t\\
    &&x_{n}(t)=\Big(\frac{5}{6}+\ldots+\frac{5}{6^{n}}\Big)t\\
\end{eqnarray*}
\textit{формально мы сделали так:} $x_{0}(t)=0,\ x_{1}(t)=Ax_{0},\ x_{2}=Ax_{1},\ldots$
Получаем следующее:
\begin{eqnarray*}
    &&x^{*}(t)=\lim\limits_{n\rightarrow \infty}x_{n}(t)
    =\lim\limits_{n\rightarrow \infty}
    \left(\sum\limits_{k=1}^{n}\frac{5}{6^{n}}\cdot t\right)
    = t\cdot 5\frac{\frac{1}{6}}{1-\frac{1}{6}}=t\\
    &&x^{*}(t)=t
\end{eqnarray*}

\paragraph{Пример 3. ДЗ}
Является ли отображение $f(x)=x^{2}$ сжимающим на
\begin{enumerate}
    \item на $\Big[0,\frac{1}{4}\Big]$
    \item на $\Big[\frac{1}{2},+\infty\Big]$
\end{enumerate}
Если является, найти неподвижную точку

\paragraph{Пример 4. ДЗ}
Есть оператор $Ax=1+\frac{1}{2}\int\limits_{0}^{1}ts^{2}x(s)\d{s}$, $A:C[0,1]\to C[0,1]$.
Является ли отображение сжимающим, если является, найти неподвижную точку

\paragraph{Замечания}
\begin{enumerate}
    \item Если $f(x)$ - отображение сжимающее, то $x^{*}$ ищется из уравнения
        \begin{eqnarray*}
            f(x*)=cx^{*},\ \ (c=1)
        \end{eqnarray*}
    \item $f(x)$ используется для функционалов
        \begin{eqnarray*}
        f:\ M\ \to\ \R
        \end{eqnarray*}
        \textit{отображение метрического пространства на числовую прямую}
    \item Отображение $A:\ M_{1}\ \to\ M_{2}$, $A$ - оператор (\textit{отображение из одного
        метрического пространства, в другое})
\end{enumerate}


\section{Вычисление нормы линейного ограниченного функционала.}
\subsection{Теория}
\paragraph{Функционал}\index{Функционал} - отображение метрического
пространства на числовую ось ($f:M\to\R$, $f$ - функционал).
\paragraph{Линейность функционала.}\index{Линейность функционала}
$f$ - \textbf{\textit{линеен}}, если
\begin{eqnarray*}
    f(\alpha x+\beta y)=\alpha f(x)+\beta f(y)
\end{eqnarray*}
\paragraph{Ограниченность функционала.}\index{Ограниченность функционала}
\paragraph{Норма функционала.}\index{Норма функционала}
\begin{eqnarray*}
    \Vert f \Vert
    =\sup_{x:\Vert x \Vert \leqslant 1}|f(x)|
    =\sup_{x\neq 0}\frac{\left|f(x)\right|}{\Vert x \Vert }
\end{eqnarray*}
\paragraph{Норма оператора.}\index{Норма оператора}
\begin{eqnarray*}
    \Vert A \Vert
    =\sup_{x:\Vert x \Vert \leqslant 1}\Vert Ax \Vert
    =\sup_{x\neq 0}\frac{\Vert Ax \Vert}{\Vert x \Vert }
\end{eqnarray*}
\subsection{Примеры}
\paragraph{Пример 1.}
\begin{eqnarray*}
&&f:l_{3}\to\R,\ \ x\in l_{3} \Rightarrow\ x=(x_{1},x_{2},\ldots)\\
&&f(x)=x_{1}
\end{eqnarray*}
Выпишем норму икса (в пространстве $l_{3}$):
\begin{eqnarray*}
    \Vert x \Vert
    =\left(\sum\limits_{i=1}^{\infty}|x_{i}|^{3}\right)^{\frac{1}{3}}
\end{eqnarray*}
Выпишем норму функционала:
\begin{eqnarray*}
    |f(x)|=|x_{1}|=\left(|x_{1}|^{3}\right)^{\frac{1}{3}}
    \leqslant \left(|x_{1}|^{3}+|x_{2}|^{3}\right)^{\frac{1}{3}}
    \leqslant \Vert x \Vert
\end{eqnarray*}
Воспользуемся вторым видом определения нормы:
\begin{eqnarray*}
    \Vert f \Vert =\sup_{x\neq 0}\frac{|f(x)|}{\Vert x \Vert}
    \leqslant \sup_{x\neq 0}\frac{\Vert x \Vert}{\Vert x \Vert}=1
\end{eqnarray*}
\textit{оценили норму сверху единицей}
\par Оценим норму $\Vert f \Vert$ снизу. Для этого выберем какой-то
элемент принадлежащий $l_{3}$: $x_{0}\in l_{3}$, пусть
$x_{0}=(1,0,\ldots)$. Тогда норма этого элемента $\Vert x_{0} \Vert =1$.
Находим значение модуля функционала: $|f(x_{0})|=1$
\begin{eqnarray*}
    \forall x\neq 0 \Rightarrow \sup_{x\neq 0}\frac{|f(x)|}{\Vert x \Vert}
    \geqslant \frac{|f(x_{0})|}{\Vert x_{0} \Vert}=1
\end{eqnarray*}
Сравнимаем оценки сверху и снизу:
\begin{eqnarray*}
1\leqslant \Vert f \Vert \leqslant 1
\Rightarrow\ \Vert f \Vert =1
\end{eqnarray*}

\paragraph{Пример 2.}
\begin{eqnarray*}
    &&f:C[0,1]\to\R\\
    &&f(x)=\int\limits_{0}^{1}x(t)\d{t},\ \ \ \Vert f \Vert\ -\ ?
\end{eqnarray*}
Общий алгоритм:
\begin{enumerate}
    \item Находим норму метрического пространства
    \item Находим модуль функционала
        \subitem Пытаемся затем оценить его (различные сомножители можно
        оценить числом, либо выделить где-нибудь норму по иксу)
        \subitem Ограничивая сверху, пытаемся привести к норме
        метрического пространства (скорее всего надо будет добавить
        члены от которых выражение точно меньше не станет)
        \subitem Ограничивая снизу подбираем такой элемент, чтобы оценка
        получилась точной
        \subsubitem В случае пространства $l_{n}$ можно использовать
        равенство и неравенство Гёльдера (см пример 8)
\end{enumerate}

Норма по непрерывным функциям:
\begin{eqnarray*}
    \Vert x \Vert =\max_{0\leqslant t\leqslant 1}|x(t)|
\end{eqnarray*}
Модуль функционала:
\begin{eqnarray*}
    |f(x)|=\left|\int\limits_{0}^{1}x(t)\d{t}\right|
    \leqslant\int\limits_{0}^{1}|x(t)|\d{t}
    \leqslant\int\limits_{0}^{1}\max_{0\leqslant t\leqslant 1}|x(t)|\d{t}=
\end{eqnarray*}
Т.к. $\max_{0\leqslant t\leqslant 1}|x(t)|$ это число - его можно
вынести
\begin{eqnarray*}
    =\max_{0\leqslant t\leqslant 1}|x(t)|\cdot\int\limits_{0}^{1}dt
    =\Vert x \Vert \Rightarrow \Vert f \Vert \leqslant 1
\end{eqnarray*}
\textit{ограничили сверху}
\par Ограничим снизу. Возьмем элемент:
\begin{eqnarray*}
    &&x_{0}(t)\in C[0,1],\ \ x_{0}(t)=1\\
    &&\Vert x_{0} \Vert =1\\
    &&|f(x_{0})|=\left|\int\limits_{0}^{1}x_{0}\d{t}\right|=1
    \Rightarrow\ \Vert f \Vert \geqslant 1
\end{eqnarray*}
Сравнивая оценки снизу и сверху:
\begin{eqnarray*}
1\leqslant \Vert f \Vert \leqslant 1
\Rightarrow\ \ \ \Vert f \Vert =1
\end{eqnarray*}

\paragraph{Пример 3.}
\begin{eqnarray*}
&&f:L_{2}[0,1]\to\R\\
&&f(x)=\int\limits_{0}^{1}x(t^{2})\d{t},\ \ \ \Vert f \Vert\ -\ ?
\end{eqnarray*}
\par \textbf{Решение.} Для начала преобразуем интеграл (сделаем замену
переменных)
\begin{eqnarray*}
    \int\limits_{0}^{1}x(t^{2})\d{t}
    =\left|\begin{array}{ll}
        t^{2}=\tau & t=\sqrt{\tau}\\
        0\leqslant\tau\leqslant 1 & \d{t}=\frac{1}{2\sqrt{\tau}\d{\tau}}
    \end{array} \right|
    =\int\limits_{0}^{1}x(\tau)\frac{1}{2\sqrt{\tau}}\d{\tau}
\end{eqnarray*}
\par Считаем модуль функционала
% TODO: как мы тут сделали неравенство Минковского непонятно
\begin{eqnarray*}
    &&|f(x)|=\left|\int\limits_{0}^{1}x(\tau)\frac{1}{2\sqrt{\tau}}\d{\tau}\right|
    \leqslant\textmd{нер-во Минковского}
    \leqslant\left(\int\limits_{0}^{1}
    \left|x(\tau)\right|^{2}\d{\tau}\right)^{\frac{1}{2}}\cdot
    \left(\int\limits_{0}^{1}
    \left|\frac{1}{2\sqrt{\tau}}\right|^{2}\d{\tau}\right)^{\frac{1}{2}}=\\
    &&=\Vert x \Vert
    \cdot\left(\int\limits_{0}^{1}\frac{1}{4\tau}\d{\tau}\right)^{\frac{1}{2}}
    =\Vert x \Vert
    \cdot\left(\left.\frac{1}{4}\ln{\tau}\right|^{1}_{0}\right)^{\frac{1}{2}}
        =\infty
\end{eqnarray*}
$\Rightarrow $ функционал является неограниченным

\paragraph{Пример 4. дз}
\begin{eqnarray*}
    f:l_{3}\to\R\ \ f(x)=x_{2020}-x_{2021}
\end{eqnarray*}

\paragraph{Пример 5. дз}
\begin{eqnarray*}
    f:L_{2}[0,1]\to\R\ \ f(x)=\int\limits_{0}^{1}tx(t)\d{t}
\end{eqnarray*}

\paragraph{Пример 6. дз}
\begin{eqnarray*}
    f:C[0,1]\to\R\ \ f(x)=x\left(\frac{1}{2}\right)+\int\limits_{0}^{1}x(t)\d{t}
\end{eqnarray*}

\paragraph{Пример 7. ($l_{1}$)}
\begin{eqnarray*}
    &&f(x):l_{1}\to\R\\
    &&f(x)=\sum\limits_{k=1}^{\infty}\frac{(1-(-1)^{k})(k-1)}{k}x_{k},\ \ \
    \Vert f \Vert\ -\ ?
\end{eqnarray*}
\par Записываем модуль функционала
\begin{eqnarray*}
    |f(x)|=\left|\sum\limits_{k=1}^{\infty}
    \frac{(1-(-1)^{k})(k-1)}{k}x_{k}
    \right|
    \leqslant \sum\limits_{k=1}^{\infty}\left|(1-(-1)^{k})\right|
    \cdot\left|\frac{k-1}{k}\right|\cdot|x_{k}|\leqslant
\end{eqnarray*}
первый сомножитель оцениваем сверху двойкой, второй - единицей
\begin{eqnarray*}
    \leqslant2\sum\limits_{k=1}^{\infty}|x_{k}|=2\Vert x \Vert
\end{eqnarray*}
$\Rightarrow\ \Vert f \Vert \leqslant 2\ (\Vert x \Vert \leqslant 1)$
\par Оценим норму снизу
\begin{eqnarray*}
    &&x_{0}\in l_{1},\ \ \sup_{\Vert x \Vert \leqslant 1}|f(x_{0})|
    \geqslant|f(x_{0})|\\
    &&x_{0}=(0,\ldots,0,1,0,\ldots)\ \ \textmd{(сначала $2n$ нулей)}\\
    &&\Vert x_{0} \Vert =1
\end{eqnarray*}
\par Найдем значение функционала в этой точке:
\begin{eqnarray*}
    |f(x_{0})|=|1-(-1)^{2n+1}|\cdot\left|\frac{2n+1-1}{2n+1}\right|\cdot 1
    =2\cdot\left|1-\frac{1}{2n+1}\right| \stackrel{n\to\infty}{\to}2
\end{eqnarray*}
$\Rightarrow \Vert f \Vert \geqslant 2$
\par Сравнивая оценки сверху и снизу
\begin{eqnarray*}
2\leqslant \Vert f \Vert \leqslant 2 \Rightarrow \ \ \ \Vert f \Vert =2
\end{eqnarray*}

\paragraph{Пример 8.}
\begin{eqnarray*}
    f:M\to\R;\ \ f(x)=3x_{1}-4x_{2}
\end{eqnarray*}
\begin{enumerate}
    \item $M=l_{1}$
        \subitem Оцениваем норму сверху
        \begin{eqnarray*}
            &&|f(x)|=|3x_{1}-4x_{2}|
            \leqslant 3|x_{1}|+4|x_{2}|
            \leqslant 4|x_{1}+4|x_{2}|=\\
            &&=4(|x_{1}|+4|x_{2}|)
            \leqslant 4(|x_{1}|+|x_{2}|+|x_{3}|+\ldots)
            =4\Vert x \Vert \Rightarrow \\
            &&\Rightarrow \Vert f \Vert \leqslant 4
        \end{eqnarray*}
        \subitem Ограничиваем снизу:
        \begin{eqnarray*}
            &&x_{0}=(0,-1,0,\ldots)\\
            &&|f(x_{0})|=4\Rightarrow \\
            && \Rightarrow \Vert f \Vert \geqslant 4 \Rightarrow \Vert f \Vert =4
        \end{eqnarray*}
    \item $M=l_{2}$
        \subitem Оцениваем сверху
        \begin{eqnarray*}
            &&|f(x)|=|3x_{1}-4x_{2}|
            \leqslant\\
            &&\leqslant 3|x_{1}|+4|x_{2}|\leqslant\textmd{[Коши-Буняковского]}
            \leqslant
            (3^{2}+4^{2})^{\frac{1}{2}}(|x_{1}|^{2}+|x_{2}|^{2})^{\frac{1}{2}}=\\
            &&= 5\left(|x_{1}|^{2}+|x_{2}|^{2}\right)^{\frac{1}{2}}
            \leqslant 5\Vert x \Vert \Rightarrow \\
            && \Rightarrow \Vert f \Vert \leqslant 5
        \end{eqnarray*}
        \subitem Оцениваем снизу
        \begin{eqnarray*}
            &&x_{0}=\left(\frac{3}{5},-\frac{4}{5},0,\ldots\right)\\
            &&\Vert x_{0} \Vert =
            \left(\frac{9}{25}+\frac{16}{25}\right)^{\frac{1}{2}}=1\leqslant
            1\\
            &&|f(x_{0})|=\left|3\frac{3}{5}+4\frac{4}{5}\right|=5
            \Rightarrow\\
            && \Rightarrow \Vert f \Vert \geqslant 5 \Rightarrow \Vert f
            \Vert =5
        \end{eqnarray*}
    \item $M=l_{\infty}$
        \subitem Норма метрики: $\Vert x \Vert =\sup_{k\in\N}|x_{k}|$
        \subitem Оцениваем сверху:
        \begin{eqnarray*}
            &&|f(x)|=|3x_{1}-4x_{2}|\leqslant 3|x_{1}|+4|x_{2}|\leqslant\\
            &&\leqslant \left[\begin{array}{c}
                    \Vert x \Vert \leqslant 1 \Rightarrow
                    \sup_{k}|x_{k}|\leqslant 1 \Rightarrow \\
                    \Rightarrow \forall k\ |x_{k}|\leqslant 1
            \end{array} \right] \leqslant 7\Vert x \Vert \Rightarrow \\
            && \Vert f \Vert \leqslant 7
        \end{eqnarray*}
        \subitem Оценим норму снизу
        \begin{eqnarray*}
            &&x_{0}=(1,-1,0,\ldots)\\
            &&\Vert x_{0} \Vert =1\\
            &&|f(x_{0})|=7 \Rightarrow\ \ \Vert f \Vert \geqslant 7
            \Rightarrow\ \ \ \Vert f \Vert =7
        \end{eqnarray*}
    \item $M=l_{4}$
        \subitem Оценим сверху:
        \begin{eqnarray*}
            &&|f(x)|=|3x_{1}-4x_{2}|\leqslant 3|x_{1}|+4|x_{2}|\leqslant\\
            &&\leqslant\left[\textmd{неравенство Гёльдера
                }\begin{array}{c}
            q=4 \Rightarrow \frac{1}{q}=\frac{1}{4} \Rightarrow \\
            \Rightarrow \frac{1}{p}=\frac{3}{4}\Rightarrow p=\frac{4}{3}
        \end{array} \right]\leqslant
        \left(3^{\frac{4}{3}}+4^{\frac{4}{3}}\right)^{\frac{3}{4}}
        \left(|x_{1}|^{4}+|x_{2}|^{4}\right)^{\frac{1}{4}}\leqslant\\
        &&\leqslant
        \left(3\sqrt[3]{3}+4\sqrt[3]{4}\right)^{\frac{3}{4}}\Vert x \Vert \Rightarrow \\
        && \Rightarrow \Vert f \Vert
        \leqslant \left(3\sqrt[3]{3}+4\sqrt[3]{4}\right)^{\frac{3}{4}}
        \end{eqnarray*}

        \subitem Оценим снизу:
        По обращаем неравенство Гёльдера в равенство
        \begin{eqnarray*}
            &&b_{i}=\sign{a_{i}}\cdot|a_{i}|^{p-1}\\
            &&a_{1}=3\ \ a_{2}=-4\ \ p=\frac{4}{3} \Rightarrow \\
            &&b_{1}=3^{\frac{1}{3}}\ \ b_{2}=-4^{\frac{1}{3}}\\
            &&x_{0}=\left(\sqrt[3]{3},-\sqrt[3]{4},0,\ldots\right)\\
            &&\Vert f \Vert =\sup_{x\neq 0}\frac{|f(x)|}{\Vert x \Vert }
            \geqslant \frac{|f(x_{0})|}{\Vert x_{0} \Vert}
            \Rightarrow \Vert f \Vert
            = \left(3\sqrt[3]{3}+4\sqrt[3]{4}\right)^{\frac{3}{4}}
        \end{eqnarray*}
\end{enumerate}

\paragraph{Пример 9. дз}
\begin{eqnarray*}
    f:M\to\R\ \ \ f(x)=\int\limits_{-1}^{1}tx(t)\d{t}
\end{eqnarray*}
\begin{enumerate}
    \item $M=L_{3}[-1,1]$
    \item $M=C[-1,1]$
\end{enumerate}

\paragraph{Пример 9.1}
\begin{eqnarray*}
    &&f:C[-1,1]\to\R\\
    &&f(x)=\int\limits_{-1}^{1}x(t)\d{t}-x(0)
\end{eqnarray*}
\par Ограничиваем сверху
\begin{eqnarray*}
    |f(x)|=\left|\int\limits_{-1}^{1}x(t)\d{t}-x(0)\right|
    \leqslant \left|\int\limits_{-1}^{1}x(t)\d{t}\right|+|x(0)|
    \leqslant \int\limits_{-1}^{1}|x(t)|\d{t}+|x(0)|
    \leqslant \Vert x \Vert \left(\int\limits_{-1}^{1}+1\right)
    =3\Vert x \Vert \Rightarrow \Vert f \Vert \leqslant 3
    \ \ (x:\Vert x \Vert \leqslant 1)
\end{eqnarray*}
\par Ограничиваем снизу
\begin{eqnarray*}
    x_{0}\in C[-1,1] \Rightarrow \Vert f \Vert
    =\sup_{\Vert x \Vert \leqslant 1}|f(x)|\geqslant |f(x_{0})|
\end{eqnarray*}
\par Возьмем последовательность функций
\begin{eqnarray*}
    &&x_{0n}(t)=\left\{\begin{array}{ll}
            1,&t\in\Bigg[-1;-\frac{1}{n}\Bigg)
            \cup\Bigg(\frac{1}{n},1\Bigg]\\
            -2nt-1,&t\in\Bigg[-\frac{1}{n},0\Bigg)\\
            2nt-1,&t\in\Bigg[0,\frac{1}{n}\Bigg)
    \end{array} \right.\\
    && \forall n\ \Vert x_{0n} \Vert =1
    \Rightarrow x_{0n}\in {x: \Vert x \Vert \leqslant 1}\\
    &&|f(x_{0n})=\left|\int\limits_{-1}^{1}x_{0n}(t)\d{t}-x_{0n}(0)\right|
    =\left|\int\limits_{-1}^{-\frac{1}{n}}\d{t}
    +\int\limits_{\frac{1}{n}}^{1}\d{t}
    +\int\limits_{-\frac{1}{n}}^{0}(-2nt-1)\d{t}
    +\frac{0}{\frac{1}{n}}(2nt-1)\d{t}+1\right|=\\
    &&=\left|-\frac{1}{n}+1+1-\frac{1}{n}+(-nt^{2}-t)\Big|^{0}_{-\frac{1}{n}}+(nt^{2}-t)\Big|^{\frac{1}{n}}_{0}+1\right|=\\
    &&=3-\frac{2}{n} \Rightarrow \Vert f \Vert \geqslant 3-\frac{2}{n}
    \stackrel{n\to\infty}{\to} 3
\end{eqnarray*}
\par Сравнивая оценки сверху и снизу получим: $\Vert f \Vert = 3$

\paragraph{Пример 10.}
\begin{eqnarray*}
&&f:C[-1,1]\to\R\\
&&f(x)=\int\limits_{-1}^{0}x(-t)\d{t}
-\int\limits_{0}^{1}x(t^{2})\d{t}
\end{eqnarray*}
\par Построим оценку сверху
\begin{eqnarray*}
    &&|f(x)|=\left|\int\limits_{-1}^{0}x(-t)\d{t}
    -\int\limits_{0}^{1}x(t^{2})\d{t}\right|
    =\left[\begin{array}{c}
            s=-t\\t=-s\\t^{2}=\tau
    \end{array}
    \begin{array}{c}
        t=\sqrt{\tau}\\\d{t}=-\d{s}
    \end{array}
    \begin{array}{c}
        \d{t}=\frac{1}{2\sqrt{\tau}}\d{\tau}
    \end{array} \right]
    =\left|-\int\limits_{0}^{1}x(s)\d{s}
    -\int\limits_{0}^{1}x(\tau)\frac{1}{2\sqrt{\tau}}\d{\tau}\right|\\
    &&=\left|\int\limits_{0}^{1}
    \left(1+\frac{1}{2\sqrt{\tau}}\right)x(\tau)\d\tau\right|
    \leqslant\int\limits_{0}^{1}
    \left|1+\frac{1}{2\sqrt{\tau}}\right|\cdot|x(\tau)|\d\tau
    \leqslant \Vert x \Vert
    \int\limits_{0}^{1}\left(1+\frac{1}{2\sqrt{\tau}}\right)\d\tau
    =2\Vert x \Vert \Rightarrow \Vert f \Vert \leqslant 2
\end{eqnarray*}
\par Построим оценку снизу
\begin{eqnarray*}
    x_{0}\in C[-1,1]:\Vert f \Vert
    =\sup_{\Vert x \Vert \leqslant 1}|f(x)|\geqslant |f(x_{0})|
\end{eqnarray*}
\par Рассмотрим ряд функций
\begin{eqnarray*}
    &&x_{0n}(t)=\left\{\begin{array}{ll}
            1,&t\in\Big[-1,-\frac{1}{n}\Big)\\
            -nt,&t\in\Big[-\frac{1}{n},\frac{1}{n}\Big]\\
            -1,&t\in\Big(\frac{1}{n},1\Big]
    \end{array} \right.\\
    &&\forall n\ \Vert x_{0n} \Vert =1
    \ \ x_{0n}\in\left\{x:\Vert x \Vert \leqslant 1\right\}\\
    &&|f(x_{0n})|
    =\left|\int\limits_{-1}^{-\frac{1}{n}}\d{t}
    +\int\limits_{-\frac{1}{n}}^{0}nt\d{t}
    +\int\limits_{0}^{\frac{1}{n}}nt^{2}\d{t}
    +\int\limits_{\frac{1}{n}}^{1}\d{t}\right|=\\
    &&=\left|-\frac{1}{n}+1
    +\frac{nt^{2}}{2}\Big|_{-\frac{1}{n}}^{0}
    +\frac{nt^{3}}{3}\Big|^{\frac{1}{n}}_{0}
    +1-\frac{1}{n}\right|
    =\left|2-\frac{2}{n}+0-\frac{1}{2n}+\frac{1}{3n^{2}}-0\right|=\\
    &&=2-\frac{5}{2n}+\frac{1}{3n^{2}}\ \ (n\geqslant 2)
    \stackrel{n\to\infty}{\to}2 \Rightarrow \Vert f \Vert \geqslant 2
\end{eqnarray*}
Сравнивая полученные оценки: $\Vert f \Vert =2$

\paragraph{Пример 11. дз}
\begin{eqnarray*}
    f:L_{1}[0,1]
    \ \ f(x)=\int\limits_{0}^{1}x(\sqrt{t})\d{t}
\end{eqnarray*}

\paragraph{Пример 12. дз}
\begin{eqnarray*}
    f:L_{2}[-1,1]
    \ \ f(x)=\int\limits_{-1}^{1}tx(t)\d{t}
\end{eqnarray*}

\paragraph{Пример 13. ($\R^{n}\to l_{2}$)}
\begin{eqnarray*}
    &&A:\R^{n}\to l_{2}\\
    &&A(x_{1},\ldots,x_{n})=
    \left(\frac{x_{1}}{\sqrt{1!}},\ldots, \frac{x_{n}}{\sqrt1!},
    \frac{x_{1}}{\sqrt{2!}},\ldots,\frac{x_{n}}{\sqrt{2!}},
    \frac{x_{1}}{\sqrt{3!}},\ldots,\frac{x_{n}}{\sqrt{3!}},\ldots\right)
\end{eqnarray*}
\par Оценим норму сверху
\begin{eqnarray*}
    &&x\in\R^{n} \Rightarrow \Vert x \Vert =|x|
    =(x_{1}^{2}+\ldots+x_{n}^{2})^{\frac{1}{2}}\\
    &&Ax=y\in l_{2} \Rightarrow \Vert y \Vert =\Vert Ax \Vert
    =\left(\sum\limits_{i=1}^{\infty}|y_{i}|^{2}\right)^{\frac{1}{2}}\\
    &&\Vert Ax \Vert = \left(
        \frac{x_{1}^{2}}{1!}+\ldots+\frac{x_{n}^{2}}{1!}
        +\frac{x_{2}^{2}}{2!}+\ldots+\frac{x_{n}^{2}}{2!}
    +\ldots \right)^{\frac{1}{2}}=\\
    &&=\left(\frac{1}{1!}(x_{1}^{2}+\ldots+x_{n}^{2})
    +\frac{1}{2!}(x_{1}^{2}+\ldots+x_{n}^{2})
    +\ldots\right)^{\frac{1}{2}}
    =\left((x_{1}^{2}+\ldots+x_{n}^{2})
    \left(\frac{1}{1!}+\frac{1}{2!}+\ldots\right)\right)^{\frac{1}{2}}
    =\\
    &&=\Vert x \Vert
    \left(\sum\limits_{k=1}^{\infty}\frac{1}{k!}\right)^{\frac{1}{2}}
    =\textmd{(прибавили, вычли единицу)}=\sqrt{e-1}\cdot \Vert x \Vert
    \Rightarrow \Vert A \Vert \leqslant \sqrt{e-1}
\end{eqnarray*}
\par Оценим норму снизу:
\begin{eqnarray*}
    x_{0}:\Vert x_{0} \Vert \leqslant 1
    \Rightarrow \Vert A \Vert
    =\sup_{\Vert x \Vert \leqslant 1}\Vert Ax \Vert
    \geqslant \Vert Ax_{0} \Vert
\end{eqnarray*}
Пусть:
\begin{eqnarray*}
    &&x_{0}=(0,1,0,\ldots,0),\ \ (\textmd{$n-2$ ноликов следуют за
    единицей})\\
    &&Ax_{0} \left(
    0,\frac{1}{\sqrt{1!}},0,\ldots,0,
    0,\frac{1}{\sqrt{2!}},0,\ldots,0,\ldots\right)\\
    &&\Vert Ax_{0} \Vert =\left(
        0^{2}+\frac{1}{1!}+0^{2}+\ldots+0^{2}+
        0^{2}+\frac{1}{2!}+0^{2}+\ldots+0^{2}+\ldots
    \right)^{\frac{1}{2}}
    =\left(\sum\limits_{k=1}^{\infty}\frac{1}{k!}\right)^{\frac{1}{2}}
    =\sqrt{e-1}
    \Rightarrow \Vert A \Vert \geqslant \sqrt{e-1}
\end{eqnarray*}
Сравнивая полученные оценки: $\Vert A \Vert =\sqrt{e-1}$

\paragraph{Пример 14. ($C[0,1]\to C[0,1]$)}
\begin{eqnarray*}
    A:C[0,1]\to C[0,1]
    \ \ Ax=\int\limits_{0}^{1}e^{3t-2\tau}x(\tau)\d\tau
\end{eqnarray*}
Видим что
\begin{eqnarray*}
    x\in C[0,1] \Rightarrow \Vert x \Vert
    =\max_{0\leqslant t\leqslant 1}|x(t)|
\end{eqnarray*}
Оценим сверху:
\begin{eqnarray*}
&&\Vert Ax \Vert
=\max_{0\leqslant t\leqslant 1}
\left|\int\limits_{0}^{1}e^{3t-2\tau}x(\tau)\d\tau\right|
\leqslant \max_{0\leqslant t\leqslant 1}e^{3t}
\cdot\int\limits_{0}^{1}e^{-2\tau}|x(\tau)|\d\tau
\leqslant \Vert x \Vert \cdot\max_{0\leqslant t\leqslant 1}
e^{3t}\cdot\left(-\frac{1}{2}e^{-2\tau}\Big|^{1}_{0}\right)=\\
&&=\Vert x \Vert \cdot\frac{1}{2}(e^{3}-e)
\Rightarrow \Vert A \Vert \leqslant \frac{1}{2}(e^{3}-e)
\end{eqnarray*}
Оценим норму снизу
\begin{eqnarray*}
    \displaystyle
    &&x_{0}\in C[0,1]:\Vert x_{0} \Vert \leqslant 1
    \Rightarrow \Vert A \Vert = \sup_{\Vert x \Vert \leqslant 1}\Vert Ax \Vert
    \geqslant \Vert Ax_{0} \Vert\\
    &&x_{0}(t)=1\\
    &&\Vert Ax_{0} \Vert =\max_{0\leqslant t\leqslant 1}
    \left|\int\limits_{0}^{1}e^{3t-2\tau}\cdot 1\cdot \d\tau\right|
    =\max_{0\leqslant t\leqslant 1}e^{3t}\cdot
    \left|\int\limits_{0}^{1}e^{-2\tau}\d\tau\right|
    =\max_{0\leqslant t\leqslant 1}e^{3t}\cdot
    \left|-\frac{1}{2}e^{-2\tau}\Big|^{1}_{0}\right|
    =\frac{1}{2}(e^{3}-e)\Rightarrow \\
    && \Rightarrow \Vert A \Vert \geqslant \frac{1}{2}(e^{3}-e)
\end{eqnarray*}
Сравнивая полученные оценки: $\Vert A \Vert =\frac{1}{2}(e^{3}-e)$

\paragraph{Пример 15. ($L_{2}[0,1]\to L_{2}[0,1]$)}
\begin{eqnarray*}
    A:L_{2}[0,1]\to L_{2}[0,1]
    ,\ \ Ax=\int\limits_{0}^{1}tx(\tau)\d\tau
\end{eqnarray*}
Норма икса:
\begin{eqnarray*}
    x\in L_{2}[0,1] \Rightarrow
    \Vert x \Vert =\left(\int\limits_{0}^{1}|x(t)|^{2}\d{t}\right)\frac{1}{2}
\end{eqnarray*}
Ограничим сверху
\begin{eqnarray*}
    &&\Vert Ax \Vert =
    \left(\int\limits_{0}^{1}
    \left|\int\limits_{0}^{1}tx(\tau)\d\tau\right|^{2}\d{t}\right)^{\frac{1}{2}}
    =\left(\int\limits_{0}^{1}t^{2}
    \left(\int\limits_{0}^{1}|x(\tau)|\d\tau\right)^{2}\d{t}\right)^{\frac{1}{2}}
    =\left(\int\limits_{0}^{1}t^{2}\Vert x \Vert^{2}
    \left(\int\limits_{0}^{1}\d\tau\right)^{2}\d{t}\right)^{\frac{1}{2}}=\\
    &&=\Vert x \Vert
    \cdot\left(\int\limits_{0}^{1}t^{2}\d{t}\right)^{\frac{1}{2}}
    =\Vert x \Vert
    \cdot\left(\frac{1}{3}t^{3}\Big|^{1}_{0}\right)^{\frac{1}{2}}
    =\frac{1}{3}\Vert x \Vert \Rightarrow \\
    && \Rightarrow \Vert A \Vert \leqslant \frac{1}{3}
\end{eqnarray*}
Оценим норму снизу
\begin{eqnarray*}
    &&x_{0}\in L_{2}[0,1]:\Vert x_{0} \Vert \leqslant 1
    \Rightarrow \Vert A \Vert = \sup_{\Vert x \Vert \leqslant 1}\Vert Ax \Vert
    \geqslant \Vert Ax_{0} \Vert\\
    &&x_{0}(t)=\frac{2}{3}
    ,\ \ \Vert x_{0} \Vert \leqslant 1\\
    &&\Vert Ax_{0} \Vert =
    \left(\left|\int\limits_{0}^{1}
    t\frac{\sqrt{2}}{3}\d{t}\right|^{2}\right)^{\frac{1}{2}}
    =\left(\frac{4}{9}\left(
    \frac{t^{2}}{2}\Big|^{1}_{0}\right)^{2}\right)^{\frac{1}{2}}
    =\frac{1}{3} \Rightarrow \Vert A \Vert \geqslant \frac{1}{3}
\end{eqnarray*}
Сравнивая оценки: $\Vert A \Vert =\frac{1}{3}$

\section{Ортогонализация Грама-Шмидта и разложение элементов евлидова
пространства в ряд Фурье.}
\subsection{Теория}
\begin{eqnarray*}
    &&\Delta = y - \alpha x_{1}-\beta x_{2}\\
    &&\left\{\begin{array}{l}
    \Delta \perp x_{1} \Leftrightarrow (\Delta,x_{1})=0\\
    \Delta \perp x_{2} \Leftrightarrow (\Delta,x_{2})=0
    \end{array} \right.
    \textmd{ такая система позволит найти $\alpha$ и $\beta$}\\
    &&\left\{\begin{array}{l}
            (\Delta,x_{1})
            =(y,x_{1})-\alpha(x_{1}x_{1})-\beta(x_{2}x_{1})=0\\
            (\Delta,x_{2})
            =(y,x_{2})-\alpha(x_{1}x_{2})-\beta(x_{2}x_{2})=0
    \end{array} \right.
\end{eqnarray*}
\subsection{Общий ход решения}
\begin{enumerate}
    \item Выписать скалярное произведение
        \subitem Затем подсчитать необходимые скалярные произведения
        которые входят в систему выше
    \item Подставить подсчитанные значения в систему, решить ее
\end{enumerate}

\subsection{Примеры}
\paragraph{Пример 1. ($L_{2}[0,1]$)}
\begin{eqnarray*}
    &&L_{2}[0,1]\\
    &&x_{1}(t)=1,\ \ x_{2}(t)=t\\
    &&y(t)=t^{2}
\end{eqnarray*}
Проекция $y$ на множество, заданное функциями $x_{1}$ и $x_{2}$. В
пространстве $L_{2}[0,1]$ определено скалярное произведение
\begin{eqnarray*}
    (a,b)=\int\limits_{0}^{1}a(t)b(t)\d{t}
\end{eqnarray*}
Теперь можно найти те элементы, которые нас интересуют:
\begin{eqnarray*}
    &&(x_{1}x_{1})
    =\int\limits_{0}^{1}1\cdot 1\d{t}=1\\
    &&(x_{1}x_{2})
    =\int\limits_{0}^{1}1\cdot t\d{t}
    =\frac{t^{2}}{2}\bigg|^{1}_{0}=\frac{1}{2}\\
    &&(x_{2}x_{2})
    =\int\limits_{0}^{1}t\cdot t\d{t}
    =\frac{t^{3}}{3}\bigg|^{1}_{0}=\frac{1}{3}\\
    &&(yx_{1})
    =\int\limits_{0}^{1}t^{2}\cdot 1\d{t}
    =\frac{t^{3}}{3}\bigg|^{1}_{0}=\frac{1}{3}\\
    &&(yx_{2})=\int\limits_{0}^{1}t^{2}\cdot t\d{t}
    =\frac{t^{4}}{4}\bigg|^{1}_{0}=\frac{1}{4}
\end{eqnarray*}
Теперь можно составить систему
\begin{eqnarray*}
    &&\left\{\begin{aligned}
    \frac{1}{3}-\alpha\cdot 1-\beta\cdot\frac{1}{2}=0\\
    \frac{1}{4}-\alpha\cdot\frac{1}{2}-\beta\cdot\frac{1}{3}=0
    \end{aligned} \right.\\
    &&\beta = 1,\ \ \alpha=\frac{1}{6}\\
    &&\hat{y}(t)=\alpha x_{1}(t)+\beta x_{2}(t)
    =\frac{1}{6}+t\\
    &&\Delta=y-\hat{y}=t^{2}-t-\frac{1}{6}\\
    &&|\Delta|=\left(\left(\Delta\Delta\right)\right)^{\frac{1}{2}}\\
    &&\left|\Delta\right|^{2}=\Delta^{2}
    =\int\limits_{0}^{1}(t^{2}-t-\frac{1}{6})^{2}\d{t}
    =\frac{7}{60} \Rightarrow |\Delta|=\sqrt{\frac{7}{60}}
\end{eqnarray*}
Расстояние от эл $y(t)=t^{2}$ до мн-ва, заданного ф-циями $x_{1}$ и
$x_{2}(t)$: $\sqrt{\frac{7}{60}}$

\paragraph{Пример 2. ($L_{2}[-1,1]$)}
\begin{eqnarray*}
    &&y(t)=t,\ \ x_{1}(t)=1,\ \ x_{2}(t)=t^{2}\\
    &&\hat{y}-?,\ \ |\Delta| -?
\end{eqnarray*}
Составляем систему уравнения для нашего случая:
\begin{eqnarray*}
    &&\Delta=y-\alpha\cdot 1-\beta t^{2}\\
    &&\left\{\begin{aligned}
    (\Delta,1)=(y,1)-\alpha(1,1)-\beta(t^{2},1)=0\\
    (\Delta,t^{2})=(y,t^{2})-\alpha(1,t^{2})-\beta(t^{2},t^{2})=0
    \end{aligned} \right.
\end{eqnarray*}
Находим элементы, которые входят в эту систему:
\begin{eqnarray*}
&&y=t\\
&&(y,1)=\int\limits_{-1}^{1}t\cdot 1\d{t}=0\\
&&(y,t^{2})=\int\limits_{-1}^{1}t\cdot t^{2}\d{t}=0\\
&&(1,1)=\int\limits_{-1}^{1}1\cdot 1\d{t}=2\\
&&(t^{2},1)=\int\limits_{-1}^{1}t^{2}\cdot 1\d{t}
=\frac{2}{3}\\
&&(t^{2},t^{2})=\int\limits_{-1}^{1}t^{2}t^{2}\d{t}=\frac{2}{5}
\end{eqnarray*}
Получается система:
\begin{eqnarray*}
    \left\{\begin{aligned}
    &2\alpha+\frac{2}{3}\beta=0\\
    &\frac{2}{3}\alpha+\frac{2}{5}\beta=0
    \end{aligned} \right.
    \Rightarrow \left\{\begin{aligned}
            &3\alpha+\beta=0 \\ &5\alpha + 3\beta=0
    \end{aligned} \right.
    \Rightarrow \left|\begin{array}{cc}
            3 & 1\\5&3
    \end{array} \right|=4\neq 0
\end{eqnarray*}
$\Rightarrow $ решение единственное: $\alpha=\beta=0$ $\Rightarrow $
$\hat{y}=0$

\paragraph{Пример 3. Найти расстояние до мн-ва ($L_{2}[0,1]$)}
В пр-ве $L_{2}[0,1]$ найти расстояние от $y(t)=t^{2}$ до мн-ва:
\begin{eqnarray*}
    M=\left\{x:\int\limits_{0}^{1}x(t)\d{t}=0\right\}
    \Rightarrow (x,1)=0
\end{eqnarray*}
Нарисовав рисунок, увидим что единица и $x$ образуют прямоугольник,
диагональю которого будет являтся $y(t)=t^{2}$, отсюда:
\begin{eqnarray*}
&&\alpha\cdot 1=\Delta\\
&&\left\{\begin{aligned}
&a=t^{2}-\alpha\cdot 1\\
&a\perp 1
\end{aligned} \right.\\
&&(a,1)=0=\int\limits_{0}^{1}(t^{2}-\alpha)1\d{t}=0\\
&&\int\limits_{0}^{1}t^{2}\d{t}-\alpha\int\limits_{0}^{1}1\d{t}=0\\
&&\frac{1}{3}-\alpha=0 \Rightarrow \alpha = \frac{1}{3}
\end{eqnarray*}
Теперь:
\begin{eqnarray*}
&&\Delta=\frac{1}{3}\cdot 1\\
&&\Delta^{2}=\int\limits_{0}^{1}\left(\frac{1}{3}\cdot
1\right)^{2}\d{t}=\frac{1}{9}\\
&&|\Delta|=\frac{1}{3}
\end{eqnarray*}

\paragraph{Пример 4. дз}
\begin{eqnarray*}
    L_{2}[0,\frac{\pi}{2}]
\end{eqnarray*}
\begin{enumerate}
\item
    Рассмотреть расстояние от $y(t)=\sin{t}$ до мн-ва, зад. ф-циями
    $x_{1}(t)=1$ и $x_{2}(t)=t$
\item
    от $y(t)=\cos{t}$ до $x_{1}(t)=1$ и $x_{2}(t)=t^{2}$
\item
    $L_{2}[0,1]$ в пр-ве есть система функций: $x_{1}(t)=1$,
    $x_{2}(t)=t$, $x(t)=t^{2}$ построить ортонормированную систему
\end{enumerate}



\section{Вычисление сильных и слабых пределов в нормированных
пространствах.}
\section{Вычисление равномерного, сильного и слабого предела
последовательностей операторов.}
\section{Построение сопряженного оператора.}
\section{Вычисление нормы и спектра линейного оператора.}

\section{Теория}

\paragraph{Сумма бесконечно убывающей геометрической прогрессии}
\paragraph{Сжимающее отображение}
Если отображение является сжимающим, то пределом последовательности служит
неподвижная точка и она не зависит от выбора начального приближения
\paragraph{Фундаментальная последовательность.}
$\left\{x_{n}\right\}$ является фундаментальной последовательностью, если
\begin{eqnarray*}
    \forall\varepsilon>0,\ \exists N=N(\varepsilon):\forall n,m\geqslant N
    \ \left|x_{n}-x_{m}\right|<\varepsilon
\end{eqnarray*}

\paragraph{Полное метрическое пространство.}\index{Полное метрическое пространство}
Каждая фундаментальная последовательность сходится к элементу этого же пространства.

\paragraph{Неравенство Коши-Буняковского.}\index{Неравенство
Коши-Буняковского}
\begin{eqnarray*}
    \sum\limits_{i=1}^{n}|a_{i}b_{i}|
    \leqslant \left(\sum\limits_{i=1}^{n}|a_{i}|^{2}\right)^{\frac{1}{2}}
    \left(\sum\limits_{i=1}^{n}|b_{i}|^{2}\right)^{\frac{1}{2}}
\end{eqnarray*}

\paragraph{Неравенство Гёльдера.}\index{Неравенство Гёльдера}
\begin{eqnarray*}
    \sum\limits_{i=1}^{n}|a_{i}b_{i}|
    \leqslant
    \left(\sum\limits_{i=1}^{n}|a_{i}|^{p}\right)^{\frac{1}{p}}
    \left(\sum\limits_{i=1}^{n}|b_{i}|^{q}\right)^{\frac{1}{q}},\ \
    \frac{1}{p}+\frac{1}{q}=1
\end{eqnarray*}

\paragraph{Равенство Гёльдера.}\index{Равенство Гёльдера}
Обращается в равенство при $b_{i}=\sign{a_{i}}\cdot |a_{i}|^{p-1}$

\paragraph{Сумма бесконечно убывающей геометрической прогрессии.}
\index{Сумма бесконечно убывающей геометрической прогрессии}
\begin{eqnarray*}
    S=\frac{a_{0}}{1-q}
\end{eqnarray*}


\subsection{Свойства модулей}
\begin{itemize}
    \item $|a+b|\leqslant |a|+|b|$
    \item $|a-b|\leqslant |a|+|b|$
    \item $\left|\int\limits
        f(x)\d{x}\right|\leqslant\int\limits|f(x)|\d{x}$
\end{itemize}

\subsection{Метрические пространства}
% в чем разница между нормой/мерой/метрикой
\begin{tabular}{l|c|c}
    Метрическое пр-во & Норма $\Vert x \Vert$ & $(a\cdot b)$ \\\hline
    $l_{\infty}$ & $\sup_{k\in\N}|x_{k}|$ \\\hline
    $l_{n}$ &
    $\left(\sum\limits_{i=1}^{\infty}x_{i}^{n}\right)^{\frac{1}{n}}$ \\\hline
    $L_{n}[a,b]$ &
    $\left(\int\limits_{a}^{b}|f(x)|^{p}\d{x}\right)^{\frac{1}{p}}$ &
    $\int\limits_{a}^{b}a(t)b(t)\d{t}$\\\hline
    $C[0,1]$ & $\max_{0\leqslant t\leqslant 1}|x(t)|$ \\\hline
    $\R^{n}$ & $\left(\sum\limits_{i=1}^{n}x_{i}^{2}\right)^{\frac{1}{2}}$
\end{tabular}

\subsection{Формула Рисса}
\subsubsection{$l_{p}$}
Теорема
\begin{eqnarray*}
    &&\forall f\in l^{*}_{p},\ p\in(1,\infty)\\
    &&\exists \widetilde{f}\in l_{q}\ \left(\frac{1}{p}+\frac{1}{q}=1\right)\\
    &&f(x)=\sum\limits_{k=1}^{\infty}\widetilde{f}_{k}x_{k}\\
    &&\Vert f \Vert =\Vert \widetilde{f} \Vert
\end{eqnarray*}

Пример
\begin{eqnarray*}
    &&f:l_{3}\to\R,\ \ f(x)=x_{1}+x_{2}\\
    &&\widetilde{f}=(1,1,0,0,\ldots)\\
    &&p=3,\ \ \ \frac{1}{q}+\frac{1}{3}=1\Rightarrow q=\frac{3}{2}\\
    &&\Vert f \Vert =\left(1^{\frac{3}{2}+1^{\frac{3}{2}}}\right)^{\frac{2}{3}}
    =\sqrt[3]{4}
\end{eqnarray*}

Другой пример
\begin{eqnarray*}
    &&f:l_{1}\to\R
    ,\ \ \ f(x)=\sum\limits_{k=1}^{\infty}\frac{2-k}{3+k}x_{k}\\
    &&\widetilde{f}=\left(\frac{2-1}{3+1},\frac{2-2}{3+2},\ldots\right)\\
    &&\Vert f \Vert =\sup_{k}\left|\frac{2-k}{3+k}\right|
    =\sup_{k}\left|\frac{5-(3+k)}{3+k}\right|
    =\sup_{k}\left|\frac{5}{3+k}-1\right|=1
\end{eqnarray*}

\subsubsection{$L_{p}$}
Теорема
\begin{eqnarray*}
    &&\forall f \in L_{p}^{*}([a,b]),\ \ p\in(1,\infty)\\
    &&\exists \widetilde{f}\in L_{q}([a,b]),\ \ \frac{1}{p}+\frac{1}{q}=1\\
    &&f(x)=\int\limits_{[a,b]}\widetilde{f}x\d\mu
    \ \ \ \Vert f \Vert =\Vert \widetilde{f} \Vert
\end{eqnarray*}
Пример
\begin{eqnarray*}
    &&f(x)=\int\limits_{[0,1]}x(\sqrt{t})\d{t},\ \ x\in L_{2}[0,1]\\
    &&\sqrt{t}=u,\ t=u^{2},\ \d{t}=2u\d{u}\\
    &&f(x)=\int\limits_{[0,1]}2ux(u)\d{u},\ \ \widetilde{f}(u)=2u\\
    &&p=2 \Rightarrow q=2\\
    &&\Vert f \Vert =\Vert \widetilde{f} \Vert
    =\sqrt{\int\limits_{[0,1]}(2u)^{2}\d{u}}=\frac{2}{3}
\end{eqnarray*}
Теорема для $L_{1}$
Пример
\begin{eqnarray*}
    &&f(x)=\int\limits_{[0,1]}x(\sqrt{t})\d{t}
    -\int\limits_{[0,1]}x(\sqrt[3]{t})\d{t},
    \ \ \ x\in L_{1}([0,1])\\
    &&\sqrt{t}=u\ \ t=u^{2}\ \ \d{t}=2u\d{u}\\
    &&\sqrt[3]{t}=s\ \ t=s^{3}\ \ \d{t}=3s^{2}\d{s}\\
    &&f(x)=\int\limits_{[0,1]}(2u-3u^{2})x(u)\d{u}\\
    &&\Vert f \Vert
    =\textmd{ф-ция непрерывна, поэтому существенный супремум совпадает с
    обычным } \textmd{ess}\sup_{u\in[0,1]}|2u-3u^{2}|=\\
    &&=\sup_{u\in[0,1]}\left|2u-3u^{2}\right|=1
\end{eqnarray*}
Другой пример
\begin{eqnarray*}
    &&f(x)=\int\limits_{[-1,0]}x(t)\d{t}
    -\int\limits_{[0,1]}x(t)\d{t},
    \ \ x\in L_{4}[-1,1],\ \Vert f \Vert -?\\
    &&f(x)=\int\limits_{[-1,1]}-\sign(t)x(t)\d{t},\ \ \sign(t)=
    \left\{\begin{aligned}-1,t<0\\1,t>0 \end{aligned} \right.\\
    &&\Vert f \Vert
    =\left(\int\limits_{[-1,1]}|-\sign(t)|^{q}\d{t}\right)^{\frac{1}{q}}
    =2^{\frac{1}{q}}=2^{1-\frac{1}{p}}=2^{\frac{3}{4}}
\end{eqnarray*}






\section{Разное}
\paragraph{1.}
Является ли полным метрическим пространством пара:
\begin{eqnarray*}
    \Bigg(\ \R,\ \rho(x,y)=|x-y|\ \Bigg)\ ?
\end{eqnarray*}
\par Доказываем что МП является полным. Каждому элементу из $\R$ ставим в соответствие последовательность
$x+\frac{1}{n}=x_{n}\to x$ (всегда сходится к элементу метрического пространства)
\par Можно ли указать мн-во $X$ такое, что МП $(X,\rho(x,y)=|x-y|)$ не является
полным. Возьмем подмножество действительных чисел $X=(0;+\infty)$, тогда фундаментальная
последовательность $x_{n}=\frac{1}{n}\to 0\notin X$
\par Можно ли указать метрику $\rho(x,y)$ такую, что МП $(\R,\rho)$ не является полным.
\begin{eqnarray*}
    \rho(x,y)=|e^{-x}-e^{-y}|
\end{eqnarray*}
Доказываем что фундаментальная последовательность есть:
\begin{eqnarray*}
    &&\rho(n,m)=|e^{-n}-e^{-m}|\leqslant \frac{2}{e^{k}}< \varepsilon,\ (k=\min(n,m))\\
    &&k\geqslant \ln{\frac{2}{\varepsilon}}=N(\varepsilon)\\
    &&\lim\limits_{n\rightarrow \infty}x_{n}=\infty
\end{eqnarray*}
$\Rightarrow $ МП: $(\R,\rho)$ не полное

\paragraph{2. Проверка линейности функционала (непрерывные функции)}
\begin{eqnarray*}
    &&f:C[0,1]\to \R\\
    &&f(x)=\int\limits_{0}^{1}x^{2}(t)\d{t}
\end{eqnarray*}
Не линеен, т.к. после подстановки получаем:
\begin{eqnarray*}
    \int\limits_{0}^{1}(\alpha^{2}x^{2}(t)+\beta^{2}y^{2}(t)+2\alpha\beta
    x(t)y(t))\d{t}
    \neq \alpha f(x)+\beta f(y)
    =\alpha\int\limits_{0}^{1}x^{2}(t)\d{t}
    +\alpha\int\limits_{0}^{1}y^{2}(t)\d{t}
\end{eqnarray*}

\paragraph{3. Проверка линейности функционала (интегрируемые функции)}
\begin{eqnarray*}
    &&f:L_{2}[0,1]\to\R\\
    &&f(x)=\int\limits_{0}^{1}x(t)\sin^{2}t\d{t}
\end{eqnarray*}
Является линейным (после раскрытия скобок обе части равны)

\paragraph{4. Проверка линейности функционала (последовательности)}
\begin{eqnarray*}
    &&f:l_{2}[0,1]\to\R\\
    &&f(x)=\sum\limits_{k=1}^{\infty}x_{k}\sin{k}
\end{eqnarray*}
Является линейным (после раскрытия скобок обе части равны)



\end{document}
