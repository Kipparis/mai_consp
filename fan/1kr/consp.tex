\documentclass{article}[12pt]


%Russian-specific packages
%--------------------------------------
\usepackage[T2A]{fontenc}
\usepackage[utf8]{inputenc}
\usepackage[russian]{babel}
%--------------------------------------

%Hyphenation rules
%--------------------------------------
\usepackage{hyphenat}
\hyphenation{ма-те-ма-ти-ка вос-ста-нав-ли-вать}
%--------------------------------------

\setlength{\parindent}{0em} % space before par
\setlength{\parskip}{1em}   % spacing between pars

\usepackage{enumitem} % configure lists
% \setlist{noitemsep,topsep=0pt} % separation
\setlist{nosep} % separation
\setlist[itemize,1]{label=$\triangleleft$} % triangles looking to the left

% margin from paper borders
\usepackage[margin={0.8in, 0.3in}]{geometry}

\usepackage{amsmath}
\usepackage{amssymb}
\usepackage{amsfonts}
% 1 - Name
% 2 - Goes to the index
% 3 - Definition
% internal - some text about teorem
% \newenvironment{theorem}[3]{
% beginning
%     \vspace{0.5\parindent}
%     \par\index{#2}\emph{\underline{Th.} \textbf{#1} #3.}  % teorem's name
%     \par
% }{
% ending
    % \vspace{0.2\parindent}
% }

% 1 - Name
% 2 - Goes to index
% internal - Name's definition
\newenvironment{mydef}[2]{
% beginning
    \vspace{0.5\parindent}
    \par \index{#2} \emph{\underline{Def.} \textbf{#1}} -
}{
% ending
    \vspace{0.2\parindent}
}

% TODO
% \newcommand{hidefinition}

% 1 - Name
% 2 - Given
% 3 - Task
% internal - solution
\newenvironment{example}[3]{
    \vspace{0.5\parindent}
    \par\emph{\underline{Ex.}} #1
    \par \textbf{Given:} #2
    \par \textbf{Find:} #3
    \par \textbf{Solution:}
}{
    \vspace{0.5\parindent}
}

% displaystyle в каждом мас моде
\everymath{\displaystyle}

% TODO: command for drawing box around text (so I can see why some artifact are

% TODO:
% \newcommand{grad}

% indent first line after section
\usepackage{indentfirst}

\usepackage{makeidx}
\makeindex

% keep this in the very bottom
% \usepackage[pdftex]{hyperref}
% remove red boxes around links
\usepackage[pdftex,pdfborderstyle={/S/U/W 0}]{hyperref} % this disables the boxes around links



\begin{document}

\section{Вычисление меры}
При вычислении меры мн-ва $A$ пользуемся неравенством:
\begin{displaymath}
    0 \leqslant \mu(A) \leqslant \sum\limits_{n=1}^{\infty}A_{n}
\end{displaymath}
где $A_{n}$ - элемент покрытия множества. Таким образом, чтобы решить
задачу, мы придумываем покрытие нашего множества, которое должно либо
сходится к числу (как например мера прямоугольнка в плоскости - его
площадь), либо сходится к нулю, как мера прямой в плоскости, либо
сходится к бесконечности (в задачах обычно означает что неправильно
построил покрытие)
\par \textit{Пример.} Рассмотрим меру бесконечной биссектрисы на
плоскости. Введем следующее покрытие: до единицы покрываем биссектрису
квадратами со стороной $\frac{1}{n}$, до двойки покрываем биссектрису
квадратами $\frac{1}{n^{2}}$. Таким образом, мера биссектрисы будет
ограниченна
\begin{displaymath}
    0 \leqslant \mu(A) \leqslant
    \lim\limits_{n\rightarrow \infty}\sum\limits_{i=1}^{\infty}\frac{1}{n^{i}}
\end{displaymath}
(в пределе мы уменьшаем размер квадратов) выражение под пределом представляет
собой бесконечно убывающую геометрическую прогрессию
\begin{displaymath}
    0 \leqslant \mu(A) \leqslant \lim\limits_{n\rightarrow
    \infty}\frac{1}{n-1}
\end{displaymath}
отсюда следует что
\begin{displaymath}
    \mu(A)=0
\end{displaymath}

\section{Вычисление интегралов}
\subsection{Интеграл Лебега}
Если функция $f$ простая - разбиваем пространство на подпространства,
где подынтегральная функция принимает фиксированное значение.

\par \textit{Формальное определение}
\begin{displaymath}
    \int\limits_{A}f(x)d\mu=\sum\limits_{i=1}^{\infty}y_{i}\mu(A_{i})
\end{displaymath}
где $y_{i}$ - значение функции на выбранном интервале, $\mu(A_{i})$ -
мера выбранного интервала.

\par \textit{Пример.}
\begin{displaymath}
    \int\limits_{[-2;+\infty)}2^{-[x]}dx
    =\mu{[-2;-1)}\cdot2^{2} + \mu{[-1;0)}\cdot2^{1}
    + \sum\limits_{n=0}^{\infty}\mu{[n;n+1)}\cdot 2^{-n}
    =4 + 2 + \sum\limits_{n=0}^{\infty}\frac{1}{2^{n}}
    =7 + \sum\limits_{n=1}^{\infty}\frac{1}{2^{n}}
\end{displaymath}
Вычисляем сумму бесконечно убывающей геометрической прогрессии
\begin{displaymath}
    \int\limits_{[-2;+\infty)}2^{-[x]}dx
    = 7 + \frac{\frac{1}{2}}{1-\frac{1}{2}}=8
\end{displaymath}

\par При решении задач с \textit{Канторовой лестницей} полезно знать,
что:
\begin{itemize}
    \item $A_{nk}=\left(\frac{3^{k}-2}{3^{n}};\frac{3^{k}-1}{3^{n}}\right)$
    \item $x_{0}\in A_{nk} \Rightarrow C(x_{0})=\frac{2k-1}{2^{n}}$
    \item $\mu(A_{nk})=\frac{1}{3^{n}}$
    \item $n=1..\infty,\ k=1..2^{n-1}$
\end{itemize}

\subsection{Интеграл Лебега-Стилтьеса}
Мера вычисляется по следующим правилам:
\begin{itemize}
    \item Для отрезка: $\mu_{F}[a,b)=F(b)-F(a)$
    \item Для точки: $\mu(\{c\})=F(c+0)-F(c)$
        \subitem Если в т. $c$ нет разрыва, то $\mu(\{c\})=0$
\end{itemize}
\par Интеграл формально
\begin{displaymath}
    \int\limits_{A}x(t)d\mu_{F}=\int\limits_{A}x(t)dF(t)
\end{displaymath}

\par \textit{Функция Хевисайда} $\chi(t)$
\begin{displaymath}
    \chi(t)=\left\{\begin{array}{ll}
    0, & t \leqslant 0 \\
    1, & t > 0
    \end{array}\right.
\end{displaymath}

\par \textit{Пример.} Вычислить
$\int\limits_{R}x(t)d_{\chi(t-a)}=\int\limits_{R}x(t)d\chi(t-a)$
\par Решение:
\begin{eqnarray*}
&&\int\limits_{R}x(t)d\chi(t-a)\stackrel{\text{разбиваем в точке разрыва на области}}{=}
\\&&=\int\limits_{(-\infty;a)}x(t)d\chi(t-a)+\int\limits_{\{a\}}x(t)d\chi(t-a)
+\int\limits_{(a;+\infty)}x(t)d\chi(t-a)
\end{eqnarray*}
Первое слагаемое:
\begin{displaymath}
    \mu_{\chi}(-\infty,a)=\chi(a-a)-\chi(-\infty)=\chi(0)-\chi(-\infty)=0
\end{displaymath}
Третье слагаемое:
\begin{displaymath}
    \mu_{\chi}(a,+\infty)=\chi(\infty)-\chi(a-a+0)=\chi(\infty)-\chi(0+0)=1-1=0
\end{displaymath}
Остается:
\begin{eqnarray*}
    &&\int\limits_{R}x(t)d_{\chi(t-a)}=\int\limits_{\{a\}}x(t)d\chi(t-a)=x(a)\mu_{\chi}(\{a\})
    =\\&&=x(a)\left(\chi(a-a+0)-\chi(a-a)\right)=x(a)(1-0)=x(a)
\end{eqnarray*}





\par Общие советы:
\begin{enumerate}
    \item Разбей на области, с границами в точках разрыва дифференциала
    \item Ищи области в которых выражение под дифференциалом не
        изменяется - эти области можно приравнять нулю
\end{enumerate}

\end{document}
