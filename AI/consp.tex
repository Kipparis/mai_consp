% открыть любую книгу по математике и улучшить шрифт в моем конспекте
\documentclass[14pt]{extarticle}

\usepackage{tikz}
\usepackage{graphicx}
\graphicspath{ {./pic/} }


%Russian-specific packages
%--------------------------------------
\usepackage[T2A]{fontenc}
\usepackage[utf8]{inputenc}
\usepackage[russian]{babel}
%--------------------------------------

%Hyphenation rules
%--------------------------------------
\usepackage{hyphenat}
\hyphenation{ма-те-ма-ти-ка вос-ста-нав-ли-вать}
%--------------------------------------

\setlength{\parindent}{0em} % space before par
\setlength{\parskip}{1em}   % spacing between pars

\usepackage{enumitem} % configure lists
% \setlist{noitemsep,topsep=0pt} % separation
\setlist{nosep} % separation
\setlist[itemize,1]{label=$\triangleleft$} % triangles looking to the left

% margin from paper borders
\usepackage[margin={0.8in, 0.3in}]{geometry}

\usepackage{amsmath}
\usepackage{amssymb}
\usepackage{amsfonts}
% 1 - Name
% 2 - Goes to the index
% 3 - Definition
% internal - some text about teorem
% \newenvironment{theorem}[3]{
% beginning
%     \vspace{0.5\parindent}
%     \par\index{#2}\emph{\underline{Th.} \textbf{#1} #3.}  % teorem's name
%     \par
% }{
% ending
    % \vspace{0.2\parindent}
% }

% 1 - Name
% 2 - Goes to index
% internal - Name's definition
\newenvironment{mydef}[2]{
% beginning
    \vspace{0.5\parindent}
    \par \index{#2} \emph{\underline{Def.} \textbf{#1}} -
}{
% ending
    \vspace{0.2\parindent}
}

% TODO
% \newcommand{hidefinition}

% 1 - Name
% 2 - Given
% 3 - Task
% internal - solution
\newenvironment{example}[3]{
    \vspace{0.5\parindent}
    \par\emph{\underline{Ex.}} #1
    \par \textbf{Given:} #2
    \par \textbf{Find:} #3
    \par \textbf{Solution:}
}{
    \vspace{0.5\parindent}
}

% displaystyle в каждом мас моде
\everymath{\displaystyle}

% vertical space for mathmode
\setlength{\abovedisplayskip}{0pt}
\setlength{\belowdisplayskip}{0pt}
\setlength{\abovedisplayshortskip}{0pt}
\setlength{\belowdisplayshortskip}{0pt}

% TODO: command for drawing box around text (so I can see why some artifact are

% TODO:
% \newcommand{grad}

% indent first line after section
\usepackage{indentfirst}

\usepackage{makeidx}
\makeindex

% keep this in the very bottom
% \usepackage[pdftex]{hyperref}
% remove red boxes around links
\usepackage[pdftex,pdfborderstyle={/S/U/W 0}]{hyperref} % this disables the boxes around links


% \setmainfont{SansForgetica}[
%   Path=./,
%   Extension=.otf,
%   UprightFont=*-Regular,
%   BoldFont=*-Regular,
%   BoldFeatures={FakeBold=3},
%   ItalicFont=*-Regular,
%   ItalicFeatures={FakeSlant=0.3},
%   BoldItalicFont=*-Regular,
%   BoldItalicFeatures={FakeBold=3,FakeSlant=0.3},
% ]
\begin{document}

\tableofcontents\newpage

\section{Общие определения}
\par\textit{Искусственный интеллект}\index{Искусственный интеллект} - раздел
информатики, изучающий реализацию в ЭВМ человеческих способов рассуждения и
решения задач.

\par\textit{Поиск в пространстве состояний}\index{Поиск в пространстве
состояний} - группа математических методов, предназначенных для решения задач
искусственного интеллекта. Методы поиска осуществляют последовательный просмотр
конфигураций или состояний задачи с целью обнаружения целевого состояния,
имеющего заданные характеристики или удовлетворяющего некоторому критерию.

\par\textit{Сильный ИИ}\index{Сильный ИИ} - интеллектуальное существо которое
будет само себя осознавать.

\par\textit{Слабый ИИ}\index{Слабый ИИ} - в какой-то задаче применяем ИИ (не
претендуем на самосознающее создание).

\section{Два (и более) подхода к созданию интеллекта}
\par Нисходящий (top-down, символьный, семиотический): многие экспертные
системы, диагностические системы (много if-else). Моделирование в компьютере
процессов рассуждений человека.
\par Восходящий (bottom-up, нейро, биологический): пытаемся смоделировать самые низкоуровневые процессы
(как взаимодействуют нейроны). В модели, на вход нейрону поступает множество
значений с коэффициентами, которые он суммирует и пропускает через функцию
активации. Обычно выстраиваются в слои (больше слоев - сложнее обучение)
\par Эволюционный (генетический): выберем несколько решений задачи и будем
скрещивать лучшие решения (решение задачи оптимизации, генетические алгоритмы).
\par Эмерджентный (синергетический): берем много объектов с простым поведением и
ожидаем от них чего-то более интересного (многоагентные системы, семейства
ботов)

\begin{center}
    \begin{tabularx}{\textwidth}{|l|X|}
        \multicolumn{2}{c}{Сравнение двух подходов}\\\hline
        Явное представление знаний & Нейросети \\\hline
        Интерпретируемый & Неинтерпретируемый \\\hline
        Необходимы эксперты & Необходимы данные (размеченные) \\\hline
        Понятное внесение изменений & Для внесения изменений необходимо
        повторное обучение \\\hline
    \end{tabularx}
\end{center}

\section{Машинное обучение}
\par Подаем данные на вход компьютеру, дальше модель что-то с ними делает (надо
сделать сначала Feature Extraction). В этом и есть ключевая особенность: в ML мы
сами выбираем и извлекаем фичи (извлечение знаний из данных, knowledge mining,
построение моделей, способных решать задачи).
\par При предсказывании:
\begin{enumerate}
    \item Строим модель: функцию которая принимает значение параметров и
        возвращает предсказание
    \item Строим функцию ошибок: функция принимает реальное и предсказанное
        значение. Возвращает значение, насколько сильно мы ошиблись
    \item Обучение: подбираем коэффициенты \textit{модели} чтобы уменьшить
        \textit{функцию ошибок}
\end{enumerate}

\begin{center}
    \begin{tabularx}{\textwidth}{|X|X|}
        \multicolumn{2}{c}{Нейросети vs. Машинное обучение}\\\hline
        Классическое машинное обучение & Нейронные сети \\\hline
        Небольшое количество параметров & Огромное количество параметров \\\hline
        Сравнительно немного обучающих примеров & Очень много обучающих
        примеров \\\hline
        Необходим ручной подбор фич & Автоматически подбирают фичи, исходные
        данные на вход \\\hline
        Сравнительно быстрое обучение & Очень ресурсоемнок обучение (на GPU) \\\hline
        Разделяющие поверхности специфической формы & Произвольные разделяющие
        поверхности \\\hline
    \end{tabularx}
\end{center}



\end{document}

