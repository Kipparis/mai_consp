\documentclass[12pt]{extarticle}

\usepackage{mathrsfs}

%Russian-specific packages
%--------------------------------------
\usepackage[T2A]{fontenc}
\usepackage[utf8]{inputenc}
\usepackage[russian]{babel}
%--------------------------------------

%Hyphenation rules
%--------------------------------------
\usepackage{hyphenat}
\hyphenation{ма-те-ма-ти-ка вос-ста-нав-ли-вать}
%--------------------------------------

\setlength{\parindent}{0em} % space before par
\setlength{\parskip}{1em}   % spacing between pars

\usepackage{enumitem} % configure lists
% \setlist{noitemsep,topsep=0pt} % separation
\setlist{nosep} % separation
\setlist[itemize,1]{label=$\triangleleft$} % triangles looking to the left

% margin from paper borders
\usepackage[margin={0.8in, 0.3in}]{geometry}

\usepackage{amsmath}
\usepackage{amssymb}
\usepackage{amsfonts}
% 1 - Name
% 2 - Goes to the index
% 3 - Definition
% internal - some text about teorem
% \newenvironment{theorem}[3]{
% beginning
%     \vspace{0.5\parindent}
%     \par\index{#2}\emph{\underline{Th.} \textbf{#1} #3.}  % teorem's name
%     \par
% }{
% ending
    % \vspace{0.2\parindent}
% }

% 1 - Name
% 2 - Goes to index
% internal - Name's definition
\newenvironment{mydef}[2]{
% beginning
    \vspace{0.5\parindent}
    \par \index{#2} \emph{\underline{Def.} \textbf{#1}} -
}{
% ending
    \vspace{0.2\parindent}
}

% TODO
% \newcommand{hidefinition}

% 1 - Name
% 2 - Given
% 3 - Task
% internal - solution
\newenvironment{example}[3]{
    \vspace{0.5\parindent}
    \par\emph{\underline{Ex.}} #1
    \par \textbf{Given:} #2
    \par \textbf{Find:} #3
    \par \textbf{Solution:}
}{
    \vspace{0.5\parindent}
}

% displaystyle в каждом мас моде
\everymath{\displaystyle}

% vertical space for mathmode
\setlength{\abovedisplayskip}{0pt}
\setlength{\belowdisplayskip}{0pt}
\setlength{\abovedisplayshortskip}{0pt}
\setlength{\belowdisplayshortskip}{0pt}

% TODO: command for drawing box around text (so I can see why some artifact are

% TODO:
% \newcommand{grad}

% indent first line after section
\usepackage{indentfirst}

\usepackage{makeidx}
\makeindex

% keep this in the very bottom
% \usepackage[pdftex]{hyperref}
% remove red boxes around links
\usepackage[pdftex,pdfborderstyle={/S/U/W 0}]{hyperref} % this disables the boxes around links


\begin{document}

% \tableofcontents \pagebreak
% TODO: сделать предметный указатель по буквам (заглавная буква, потом
% слова)
\printindex \pagebreak

\section{Электростатика. Постоянный ток}
Скалярная величина, равная отношению потенциальной энергии пробного
заряда $q'$ в электрическом поле заряда $q$ к величине пробного заряда
называется \textbf{\textit{потенциалом}}\index{Потенциал}
\begin{eqnarray*}
\varphi=\frac{W}{q'}
\end{eqnarray*}

\par Вектор, равный отношению силы $\vec{F}$, с которой заряд $q$ действует
на малый положительный точечный заряд $q'$ (так называемый
\textit{пробный заряд}), помещенный в некоторую точку пространства, к
величине пробного заряда
\begin{eqnarray*}
\vec{E}=\frac{\vec{F}}{q'}
\end{eqnarray*}
называется \textit{напряженностью электрического поля, создаваемого
зарядом $q$}\index{Напряженность электрического поля} в данной точке

\paragraph{Теорема Гаусса для электрического поля.} \index{Теорема Гаусса
для электрического поля}
Поток вектора напряженности электрического поля неподвижных зарядов
через произвольную замкнутую поверхность равен алгебраической сумме
зарядов, заключенных внутри этой поверхности, деленной на
$\varepsilon_{0}$:
\begin{eqnarray*}
    \Phi=\oint\limits_{S}(\vec{E},d\vec{S})
    =\oint\limits_{S}E_{n}dS
    =\frac{1}{\varepsilon_{0}}\int\limits_{i=1}^{N}q_{i}
\end{eqnarray*}
\paragraph{Теорема Гаусса в дифференциальной форме:} дивергенция вектора
напряженности электрического поля неподвижных зарядов в некоторой точке
равна объемной плотности заряда в той же точке, деленной на
$\varepsilon_{0}$ (\textit{заряды являются источниками электрического
поля})
\begin{eqnarray*}
    \textmd{div}\,{\vec{E}}=\frac{\rho}{\varepsilon_{0}}
\end{eqnarray*}

\paragraph{Теорема Гаусса для диполей.}\index{Теорема Гаусса для диполей}
Поток $\Phi_{D}$ вектора электрического смещения $\overline{D}$ через
замкнутую поверхность равен алгебраической сумме сторонних зарядов,
(свободных)
заключенных внутри этой поверхности:
\begin{eqnarray*}
    \Phi_{D}=\sum\limits_{i}q_{i}
    \ \ \textmd{или}
    \ \ \Phi_{D}=\oint\limits\vec{D}\cdot d\vec{S}
    =\int\limits\rho_{\textmd{св}}dV
\end{eqnarray*}

\paragraph{Связанные заряды.}\index{Связанные заряды}
Связанные заряды. В результате процесса поляризации в объеме (или на
поверхности) диэлектрика возникают нескомпенсированные заряды, которые
называются поляризационными, или связанными. Частицы, обладающие этими
зарядами, входят в состав молекул и под действием внешнего
электрического поля смещаются из своих положений равновесия, не покидая
молекулы, в состав которой они входят. Связанные заряды характеризуют
поверхностной плотностью.

\paragraph{Сторонние заряды.}\index{Сторонние заряды}
Заряды в диэлектрике, не входящие в состав его атомов и молекул,
называются сторонними зарядами

\paragraph{Теорема Гаусса для магнитного поля.}\index{Теорема Гаусса для
магнитного поля}
Поток вектора $B$ через любую замкнутую поверхность равен нулю:
\begin{eqnarray*}
\oint\limits BdS=0
\end{eqnarray*}
или в дифференциальной форме
\begin{eqnarray*}
\nabla\cdot B=0
\end{eqnarray*}
справа стоит $0$ потому что линии магнитной индукции не имеют ни начала,
ни конца. Поэтому число линий вектора B, выходящих из любого объема,
ограниченного замкнутой поверхностью $S$, всегда равно числу линий
входящих в этот объем. \textit{Магнитное поле} является полностью
вихревым.

\paragraph{Сила тока.}\index{Сила тока} Она равна суммарному заряду,
переносимому через некоторую воображаемую поверхность $S$ за единицу
времени.

\paragraph{Плотность тока.}\index{Плотность тока} Модуль этого вектора
численно равен количеству тока, которое проходит через элементарную
площадку.

\paragraph{Теорема Гельмгольца.}\index{Теорема Гельмгольца}
Если дивергенция и рото векторного поля $F(r)$ определены в каждой точке
конечной открытой области $V$ пространства, то всюду в $V$ функция может
быть представлена в виде суммы безвихревого поля $F_{1}(r)$ и
соленоидального поля $F_{2}(r)$


\paragraph{Вектор электрического смещения.}\index{Вектор электрического
смещения}
\begin{eqnarray*}
    &&\vec{D}=\varepsilon_{0}\vec{E}+\vec{P}\\
    &&\oint\limits_{S}\vec{D}d\vec{S}=\int\limits_{i=1}^{N}q_{i}
\end{eqnarray*}
поле этого вектора определяется только наличием свободных зарядов.
\par Векторная величина, равная сумме вектора напряжённости электрического
поля и вектора поляризации.


\paragraph{ЭДС.}\index{ЭДС} Величина, равная работе сторонних сил над
единичным положительным зарядом, называется \textit{электродвижущей
силой} (\textit{ЭДС}), действующей в контуре или на участке цепи
\begin{eqnarray*}
    \mathscr{E}=\frac{A_{\textmd{стор}}}{q}
\end{eqnarray*}

\paragraph{Электроемкость.}\index{Электроемкость} Опыт показывает, что
потенциал проводника пропорционален находящемуся на нем заряду:
\begin{eqnarray*}
q=C\varphi
\end{eqnarray*}
где коэффициент пропорциональности $C$ называют
\textit{электроемкостью} (или просто \textit{емкостью}) проводника.

\paragraph{Падение напряжения.}\index{Падение напряжения} Работа,
совершаемая электрической или сторонними силами при перемещении
единичного положительного заряда:
\begin{eqnarray*}
    U=\frac{A_{a-b}}{q}=(\varphi_{a}-\varphi_{b})+\mathscr{E}
\end{eqnarray*}

\paragraph{Локальный закон Ома.}\index{Локальный закон Ома}
\begin{eqnarray*}
jdS=\frac{EdldS}{\rho dl}
\end{eqnarray*}
где $j=\sigma E$, где $\sigma=\frac{1}{\rho}$ - удельная электрическая
проводимость проводника, единицей измерения которой является сименс на
метр [См/м]

\paragraph{Интегральный закон Ома.}\index{Интегральный закон Ома}
\begin{eqnarray*}
    \int\limits_{1}^{2}\vec{E}\cdot d\vec{l}
    =I\int\limits_{1}^{2}\rho\frac{dl}{S}
\end{eqnarray*}

\paragraph{Электрический момент диполя.}\index{Электрический момент
диполя}
\begin{eqnarray*}
p=ql
\end{eqnarray*}



\section{Магнетизм. Переменный ток}
\paragraph{Напряженность магнитного поля.}\index{Напряженность
магнитного поля} $\vec{H}$

\paragraph{Индукция магнитного поля.}\index{Индукция магнитного поля}
$\vec{B}$

\paragraph{Сила Лоренца.}
В случае, когда одновременно существуют электрическое и магнитное поле,
сила, действующая на движущуюся заряженную частицу - \textit{сила
Лоренца} \index{Сила Лоренца}
\begin{eqnarray*}
    \vec{F}_{\textmd{л}}=q[\vec{v},\vec{B}]+q\vec{E}
\end{eqnarray*}

\paragraph{Вектор Пойнтинга}\index{Вектор Пойнтинга} - вектор
плотности потока энергии электромагнитнго поля, компоненты которого
входят в состав компонент тензора энергии-импульса электромагнитного
поля
\begin{eqnarray*}
    S=[E\times H]
\end{eqnarray*}

\paragraph{Ток смещения.}\index{Ток смещения}
\begin{eqnarray*}
    \vec{j}_{\textmd{см}}=\varepsilon_{0}\frac{\partial\vec{E}}{\partial t}
\end{eqnarray*}
этот термин чисто условный, посколько ток смещения по сути - это
изменяющееся со временем электрическое полле. Из всех физических
свойств, присущих действительному току, ток смещения обладает лишь одним
- способностью создавать магнитные поля

\paragraph{Импеданс.}\index{Импеданс}
Комплексное сопротивление между двумя узлами цепи или двухполюсника для
гармонического сигнала.

\paragraph{Магнитный момент.}\index{Магнитный момент}
\begin{eqnarray*}
    p_{m}=IS
\end{eqnarray*}
используется для расчета силы, действующей на элементарный контур.
Задает направление для вектора магнитной индукции.

\paragraph{Теорема о циркуляции вектора магнитной индукции.}
\index{Теорема о циркуляции вектора магнитной индукции}
\index{Закон полного тока}
Циркуляция вектора индукции магнитного поля по произвольному контуру
равна алгебраической сумме токов, охватываемых контуром, умноженной на
$\mu_{0}$:
\begin{eqnarray*}
    \oint\limits_{L}(\vec{B},d\vec{l})
    =\oint\limits_{L}B_{l}dl=\mu_{0}\sum\limits_{i}I_{i}
\end{eqnarray*}

\paragraph{Опыт Фарадея.}\index{Опыт Фарадея}
По определению Фарадея общим для этих опытов является следующее: если
поток вектора индукции, пронизывающий замкнутый, проводящий контур,
меняется, то в контуре возникает электрический ток.
\par Это явление называют явлением электромагнитной индукции, а ток –
индукционным. При этом явление совершенно не зависит от способа
изменения потока вектора магнитной индукции.

\paragraph{Теорема Гаусса для диэлектриков в дифференциальной форме.}
\index{Теорема Гаусса для диэлектриков в дифференциальной форме}
Дивергенция вектора электрического смещения в некоторой точке равна
объемной плотности сторонниз зарядов в той же точке
\begin{eqnarray*}
\textmd{div}\,\vec{D}=\rho
\end{eqnarray*}


\section{Волновая оптика}
Электромагнитная волна переносит энергию в направлении своего
распространения. Энергия, переносимая волной через единицу площади в
единицу времени - \textit{интенсивность} волны\index{Интенсивность волны}:
\begin{eqnarray*}
    I=\frac{1}{2}\sqrt{\frac{\varepsilon\varepsilon_{0}}{\mu\mu_{0}}A^{2}}
\end{eqnarray*}

\paragraph{Интерференция.}\index{Интерференция}
Независимые источники света излучают волны с быстро изменяющейся
разностью фаз. Если свет от таких источников направить на экран, то в
результате наложения волн вся поверхность экрана будет равномерно
совещена с некоторой средней интенсивностью. Если же одну волну
каким-либо образом разделить на две или более, то при наложении
полученных таким образом волн разные точки экрана будут освещены с
разной интенсивностью. Явление увеличения или уменьшения интенсивности
света при наложении нескольких волн называется \textbf{интерференцией}.

\paragraph{Дифракция.}\index{Дифракция}
Дифракция - отклонение света от пряммолинейного распространения в среде
с резкими неоднородностями, что связано с отклонениями от законов
геометрической оптики. Как показывает опыт, за непрозрачной преградой с
отверстием (или щелью) свет распространяется по все направлениям. Это
приводит к огибанию волнами препятствий и проникновению света в область
геометрической тени, которое тем существеннее, чем меньше размеры
отверстий.

\paragraph{Поляризация.}\index{Поляризация}
Векторы $\vec{E}$ и $\vec{H}$ остаются взаимноперпендикулярными в каждый
момент времени, но их направления быстро и беспорядочно меняются. Свет,
у которого направления колебаний каким-либо образом упорядочены,
называется \textit{поляризованным}, а само явление -
\textbf{поляризацией}
\begin{enumerate}
    \item Линейная
    \item Круговая
    \item Эллиптическая
\end{enumerate}


\paragraph{Длина когерентности.}\index{Длина когерентности}
Для того, чтобы в точке волны были когенертными, они должны принадлежать
одному цугу волн. Для этого необходимо, чтобы разность путей, пройденных
волнами, не превосходила длину цуга $L=c\Delta t$, которую называют
\textit{длиной когерентности} $l_{\textmd{ког}}$, а длительность цуга
$\Delta t$ - \textit{временем когерентности} $t_{\textmd{ког}}$
\par Време когерентности обратно пропорционально интервалу частот,
представленных в световой волне:
\begin{eqnarray*}
    t_{\textmd{ког}}\sim\frac{2\pi}{\Delta \omega}=\frac{1}{\Delta v}
\end{eqnarray*}

\paragraph{Закон Малюса.}\index{Закон Малюса}
Интенсивность света после прохожждения поляризатора будет равна
\begin{eqnarray*}
    I=I_{0}\cos^{2}\varphi
\end{eqnarray*}
где $\varphi$ - угол между плоскостью колебаний падающего света и
плоскостью поляризатора.

\paragraph{Поперечная волна.}\index{Поперечная волна} распространяется
по вектору Пойтинга.

\paragraph{Теорема Пойнтинга.}\index{Теорема Пойнтинга} убыль энергии за
единицу времени в данном объеме равна потоку энергии сквозь поверхность,
ограниченную этим объемом, плюс работа в единицу времени (т.е. мощность
$P$), которую поле производит над зарядами вещества внутри данного
объема.

\paragraph{Фазовая скорость.}\index{Фазовая скорость}ССкорость перемещения точки, обладающей постоянной фазой колебательного
движения в пространстве, вдоль заданного направления.

\section{Теорема Остроградского-Гаусса}
\begin{eqnarray*}
    \int\limits_{S}(\vec{a},d\vec{S})
    =\int\limits_{V}\textmd{div}\,\vec{a}\,dV
\end{eqnarray*}
где интеграл в левой части берется по произвольной замкнутой поверхности
$S$, а интеграл в правой части - по объему $V$, ограниченному этой
поверхностью. Из данного соотношения следует, что \textit{дивергенция
вектора имеет смысл источников поля}.



% что такое конденсатор


\end{document}
