\documentclass[12pt]{article}
\usepackage{science}

\usepackage{array}
\setlength\extrarowheight{2pt} % or whatever amount is appropriate

%Russian-specific packages
%--------------------------------------
\usepackage[T2A]{fontenc}
\usepackage[utf8]{inputenc}
\usepackage[russian]{babel}
%--------------------------------------

% \usepackage{bookmark}

\newcommand{\M}{\mathsf{M}}
\usepackage{amsmath}
\usepackage{amssymb}

%Hyphenation rules
%--------------------------------------
\usepackage{hyphenat}
\hyphenation{ма-те-ма-ти-ка вос-ста-нав-ли-вать про-стран-ствен-но}
\hyphenation{про-стран-ствен-но}
%--------------------------------------
\usepackage[margin=1.0in]{geometry}

\usepackage{graphicx}
\graphicspath{ {./images/} }

\usepackage{bookmark}

% indent first line after section
\usepackage{indentfirst}

\setlength{\emergencystretch}{10pt}
\hbadness=10000
\hfuzz=20.002pt

% \renewcommand\kappa{\text{\ae}}

\usepackage{makeidx}
\makeindex

% keep this in the very bottom
% \usepackage[pdftex]{hyperref}
\begin{document}

\tableofcontents
\newpage
\printindex

\newpage
\section{Комбинаторика}
\begin{definition}
    {Перестановки}
    {Перестановки}
    число объектов неизменно, изменяется только их порядок. Формула:
    \begin{displaymath}
        P_{n}=n!
    \end{displaymath}
\end{definition}

\begin{definition}
    {Размещения}
    {Размещения}
    из $n$ объектов выбираем $m$ различных объектов и переставляем их
    всеми возможными способами. Формула:
    \begin{displaymath}
        A_{n}^{m}=\frac{n!}{(n-m)!}
    \end{displaymath}
\end{definition}

\begin{definition}
    {Сочетания}
    {Сочетания}
    из $n$ объектов выбираем всевозможные пары по $m$ объектов (порядок
    не имеет значения). Формула:
    \begin{displaymath}
        C_{n}^{m}=\frac{n!}{m!(n-m)!}
    \end{displaymath}
\end{definition}

\newpage
\section{Формулы}
\begin{theorem}
    {Сложение вероятностей}
    {Сложение вероятностей}
    {}
    Если (попарно) несовместимы:
    \begin{displaymath}
        P\left(\sum\limits_{i=1}^{n}A_{i}\right)=\sum\limits_{i=1}^{n}P\left(A_{i}\right)
    \end{displaymath}
    Иначе:
    % TODO: вставить полную формулу
    \begin{displaymath}
        P(A+B)=P(A)+P(B)-P(AB)
    \end{displaymath}
\end{theorem}

\begin{theorem}
    {Умножение вероятностей}
    {Умножение вероятностей}
    {}
    Если (попарно) независимы:
    \begin{displaymath}
        P\left(\prod\limits_{i=1}^{n}A_{i}\right)=\prod\limits_{i=1}^{n}P\left(A_{i}\right)
    \end{displaymath}
    Иначе:
    % TODO: формула условной вероятности полная
    \begin{displaymath}
        P(A\cdot B)=P(A)\cdot P(B|A)
    \end{displaymath}
\end{theorem}

\begin{theorem}
    {Формула полной вероятности}
    {Формула полной веростности}
    {}
    \begin{displaymath}
        P(A)=\sum\limits_{k=1}^{n}P(H_{k})\cdot P(A|H_{k})
    \end{displaymath}
    где $H_{1},H_{2},\ldots,H_{n}$ - полная группа гипотез.
\end{theorem}

\begin{theorem}
    {Формула Байеса.}
    {Формула Байеса.}
    {}
    \begin{displaymath}
        P(H_{m}|A)=\frac{P(H_{m})\cdot P(A|H_{m})}{P(A)}
        =\frac{P(H_{m}))\cdot
        P(A|H_{m})}{\sum\limits_{k=1}^{n}P(H_{k})\cdot P(A|H_{k})}
    \end{displaymath}
\end{theorem}

\newpage
\section{Общие сведения}
\begin{definition}
    {Функция распределения $F(x)$}
    {Функция распределения}
    функция, которая определяет вероятность того, что значение окажется
    меньшим или равным чем то, что в аргументе
    \begin{displaymath}
        F(x)=f(\xi<x)
    \end{displaymath}
\end{definition}

\begin{definition}
    {Закон распределения}
    {Закон распределения}
    таблица, в верхней строке которой перечислены все значения величины
    $\xi$, а в нижней строке соответствующие им вероятности
    \begin{displaymath}
        p_{n}=P(\xi=x_{n})
    \end{displaymath}
\end{definition}

\subsection{Математическое ожидание}
\begin{definition}
    {Математическое ожидание}
    {Математическое ожидание}
    Для дискретных или счетных событий:
    \begin{displaymath}
        M(\xi)=\sum\limits_{k}x_{k}p_{k}
        =\sum\limits_{k}g(x_{k})p_{k}
    \end{displaymath}
    Для непрерывных:
    \begin{displaymath}
        M(\xi)=\int\limits_{-\infty}^{+\infty}xf_{\xi}(x)dx
    \end{displaymath}
    и
    \begin{displaymath}
        M(\xi\cdot\eta)=\int\limits_{-\infty}^{+\infty}\int\limits_{-\infty}^{+\infty}
        xyf_{\xi\eta}(x,y)dx\,dy
    \end{displaymath}
    Свойства:
    \begin{enumerate}
        \item $M(C)=C$
        \item $M(\xi+\eta)=M(\xi)+M(\eta)$
        \item $M(\lambda\xi)=\lambda M(\xi)$
    \end{enumerate}
\end{definition}

\subsection{Дисперсия}
\begin{definition}
    {Дисперсия}
    {Дисперсия}
    \begin{displaymath}
        var(\xi)=D(\xi)=M(\xi^{2})-(M(\xi))^{2}=M\left[(\xi-M\xi)^{2}\right]
    \end{displaymath}
    Свойства:
    \begin{enumerate}
        \item $D(-\xi)=D(\xi)$
        \item $D(\xi+C)=D(\xi)$
        \item $D(\lambda\xi)=\lambda^{2}D(\xi)$
        \item $D(a\xi+b\eta)=a^{2}D(\xi)+b^{2}D(\eta)+2\textmd{cov}(a\xi,b\eta)$
    \end{enumerate}
\end{definition}

\begin{definition}
    {Среднеквадратическое отклонение}
    {Среднеквадратическое отклонение}
    \begin{displaymath}
        \sigma_{\xi}=\sqrt{D(\xi)}
    \end{displaymath}
\end{definition}

\subsection{Ковариация}
\begin{definition}
    {Ковариация}
    {Ковариация}
    \begin{displaymath}
        \textmd{cov}(X,Y)=k_{XY}=M[(X-M[X])(Y-M[Y])]=M[XY]-M[X]M[Y]
    \end{displaymath}
    Свойства:
    \begin{enumerate}
        \item $\textmd{cov}(\xi,\xi)=D[\xi]$
        \item $\textmd{cov}(\xi,\eta)=\textmd{cov}(\eta,\xi)$
        \item $\textmd{cov}(a_{1}\xi+b_{1};a_{2}\eta+b_{2})=a_{1}a_{2}\textmd{cov}(\xi,\eta)$
    \end{enumerate}
\end{definition}

\subsection{Корреляция}
\begin{definition}
    {Корреляция (коэффицент корреляции)}
    {Корреляция}
    \begin{displaymath}
        \textmd{corr}(X,Y)=r_{XY}=\rho_{XY}=\textmd{cov}(X,Y)/(\sigma_{X}\sigma_{Y})
    \end{displaymath}
\end{definition}

\newpage
\section{Основные распределения}
\subsection{Биномиальное распределение}
\par Обозначение: $\xi \sim Bi(n,p)$ - число успехов в $n$ испытаниях
Бернулли с вероятностью успеха $p$ в отдельном испытании.
\par $p_{k}=P_{n}(k)=P(\xi=k)=C_{n}^{k}p^{k}q^{n-k}$.
% сделать через таблицу
\par Мат. ожидание: $M(\xi)=np$
\par Дисперсия: $D(\xi)=npq$
\par Мода: $(n+1)p-1\leqslant k^{*}\leqslant(n+1)p$
\par Оценка:
\begin{eqnarray*}
    Bi(N;\theta) \Rightarrow \widehat{\theta}_{n}=\frac{\overline{X}_{n}}{N}
\end{eqnarray*}

\par Пример: $\xi$ - количество принятых сигналов, если вероятность приема
самолетом радиосигнала при каждой передаче $p = 0,7$. $\xi \sim Bi(n,p$
\par Физический смысл: число успехов

\newpage
\subsection{Геометрическое распределение}
\par Обозначение: $\xi \sim G(p)$
\par $P(\xi=k)=(1-p)^{k-1}p$
\par Физический смысл: число опытов до 1-го успеха в схеме Бернулли
\par Мат. ожидание: $M[\xi]=\frac{1}{p}$
\par Дисперсия: $D[\xi]=\frac{1-p}{p^{2}}$
\par Мода: $k^{*}=1$
\par Пример: $\xi$ - количество изделий которое выпускает автоматическая
линия между двумя настройками, если настройка производится при любом
выпуске бракованной детали.

\newpage
\subsection{Распределение Пуассона}
\par Обозначение: $\xi \sim \textmd{П}(a) \sim Pois(a),\ a>0$
\par $P(\xi=k)=\frac{a^{k}}{k!}e^{-a},\ k=0,1,2...$
\par Физический смысл: число событий из пуассоновского потока событий,
происходящее за промежуток времени фиксированной длинны.
\par Мат. ожидание: $M[\xi]=a$
\par Дисперсия: $D[\xi]=a$
\par Мода: $k^{*} = \lfloor a\rfloor$ - целая часть
\par МП оценка: $\xi\sim\textmd{П}(\theta),\ \theta > 0,\ \
\widehat{\theta}_{n}=\overline{X}_{n}$
\par Пример

\newpage
\subsection{Экспоненциальное распределение}
\par Обозначение: $\xi \sim E(\lambda) \sim Exp(\lambda)$
\par $f_{\xi}(x)=\left\{\begin{array}{ll}
        0, & x<0 \\
        \lambda e^{-\lambda x}, & x\geqslant 0
\end{array}\right.$
\par Физический смысл: время ожидание в распределении Пуассона.
\par Мат. ожидание: $M[\xi]=\frac{1}{\lambda}$
\par Дисперсия: $D[\xi]=\frac{1}{\lambda^{2}}$
\par Функция распределения: $F_{\xi}(x)=\left\{\begin{array}{ll}
        0, & x<0 \\
        1-e^{-\lambda x}, & x\geqslant 0
\end{array}\right.$
\par МП оценка: $\widehat{\theta}_{n}=\frac{1}{\overline{X}_{n}}$

\par Пример: время между появлениями двух последовательных покупателей

\newpage
\subsection{Равномерное распределение на отрезке}
\par Обозначение: $\xi \sim R(a,b) \sim U(a,b)$
\par $f_{\xi}(x)=\left\{\begin{array}{ll}
        \frac{1}{b-a}, & a\leqslant x\leqslant b
        \\ 0, & \textmd{иначе}
\end{array}\right.$

\par Функция распределения: $F_{\xi}(x)=\left\{\begin{array}{ll}
        0, & x<a
        \\ \frac{x-a}{b-a}, & a\leqslant x\leqslant b
        \\ 1, & x>b
\end{array}\right.$
\par Мат. ожидание: $M[\xi]=\frac{a+b}{2}$
\par Дисперсия: $D[\xi]=\frac{(b-a)^{2}}{12}$
\par МП оценки: $\xi\sim R(\theta_{1},\theta_{2}),
\ \widehat{\theta_{1}}_{n} = X_{1},
\ \widehat{\theta_{2}}_{n} = X_{(n)}$
\par Свойства МП оценок:
\begin{eqnarray*}
    & & \M [\widehat{\theta_{2}}_{n}] = \frac{n}{n+1}\theta_{2}
    ,\ \ \theta_{1}=0\\
    \\& & f_{\widehat{\theta_{2}}_{n}}(x)=\frac{nx^{n-1}}{\theta_{2}^{n}}
    ,\ \ \theta_{1}=0,\ \ x\in \left[0,\theta_{2}\right]\\
\end{eqnarray*}




\newpage
\subsection{Нормальное распределение}
\par Обозначение: $\xi \sim N(m,\sigma^{2})$
\par \[
    f_{\xi}(x)=\frac{1}{\sqrt{2\pi}\sigma}e^{\frac{-(x-m)^{2}}{2\sigma^{2}}}
\]
\par Физический смысл: везде используется
\par Мат. ожидание: $M[\xi]=m$
\par Дисперсия: $D[\xi]=\sigma^{2}$
\begin{definition}
    {Функция Лапласа}
    {Функция Лапласа}
    одна из двух функций (может быть разная терминология)
    \begin{displaymath}
        \Phi_{0}(x)=\frac{1}{\sqrt{2\pi}}\int\limits_{0}^{x}e^{-\frac{t^{2}}{2}}dt
        =\Phi(x)-\frac{1}{2}
    \end{displaymath}
    или
    \begin{displaymath}
        \Phi(x)=\frac{1}{\sqrt{2\pi}}\int\limits_{-\infty}^{x}e^{-\frac{t^{2}}{2}}dt
    \end{displaymath}
    - функция распределения СВ $N(0,1)$
\end{definition}
\par Функция распределения:
$F_{\xi}(x)
=\Phi\left(\frac{x-m}{\sigma}\right)
=\frac{1}{2}+\Phi_{0}\left(\frac{x-m}{\sigma}\right)$
\par Пример

\newpage
\section{Случайные векторы}

\begin{definition}
    {Функция распределения случайных векторов}
    {Функция распределения случайных векторов}
    \begin{displaymath}
        F_{\xi_{1}\ldots \xi_{n}}(x_{1},\ldots,x_{n})
        =\int\limits_{-\infty}^{x_{1}}\ldots\int\limits_{-\infty}^{x_{n}}
        f_{\xi_{1}\ldots\xi_{n}}(t_{1},\ldots,\xi_{n})dt_{1}\ldots dt_{n}
    \end{displaymath}
\end{definition}

\par Важные свойства:
\begin{enumerate}
    \item \begin{displaymath}
            \int\limits_{-\infty}^{+\infty}f_{\xi\eta}(x,y)dy=f_{\xi}(x),
            \ \ \ \int\limits_{-\infty}^{+\infty}f_{\xi\eta}(x,y)dx=f_{\eta}(y)
        \end{displaymath}
\end{enumerate}

Ковариация, коэффицент ковариации, корелляция
Независимость

\begin{definition}
    {Коррелируемость СВ}
    {Коррелируемость СВ}
    когда выполняется равенство:
    \begin{displaymath}
        \textmd{cov}(\xi,\eta)=M(\xi\eta)-M(\xi)M(\eta)=0
    \end{displaymath}
\end{definition}

\begin{definition}
    {Независимость СВ}
    {Независимость СВ}
    когда выполняется равенство:
    \begin{displaymath}
        M(\xi\eta)=M(\xi)\cdot M(\eta)
    \end{displaymath}
    для дискретных СВ:
    \begin{displaymath}
        \forall x,y: P(\xi=x;\eta=y)=P(\xi=x)\times P(\eta=y)
    \end{displaymath}
    Если $\xi$ и $\eta$ независимы $ \Rightarrow $ некоррелируемые
\end{definition}

\begin{definition}
    {Математическое ожидание случайного вектора}
    {Математическое ожидание случайного вектора}
    вектор из математических ожиданий его компонент
    \begin{displaymath}
        \xi=\left(
            \begin{array}{c} \xi_{1} \\ \vdots \\ \xi_{n} \end{array}
        \right);
        \ \ \ M[\xi]=\left(
            \begin{array}{c} M[\xi_{1}] \\ \vdots \\ M[\xi_{n}]\end{array}
        \right)
    \end{displaymath}
\end{definition}

\begin{definition}
    {Ковариационная матрица случайного вектора}
    {Ковариационная матрица случайного вектора}
    \begin{displaymath}
        K=\|k_{ij}\|_{i,j=1}^{n},\ \ k_{ij}=\textmd{cov}(\xi_{i},\xi_{j})
    \end{displaymath}
\end{definition}

\begin{definition}
    {Корреляционная матрица случайного вектора}
    {Корреляционная матрица случайного вектора}
    \begin{displaymath}
        R=\|r_{ij}\|_{i,j=1}^{n},\ \ r_{ij}=\textmd{corr}(\xi_{i},\xi_{j})
    \end{displaymath}
\end{definition}

\newpage
\section{Условные распределения}

\newpage \section{Закон больших чисел}
\index{Центральные предельные теоремы}\textit{Центральные предельные
теоремы} - класс теорем в теории
вероятностей, утверждающих, что сумма достаточно большого количества
слабо зависимых случайных величин, имеющих примерно одинаковые масштабы
(ни одно из слагаемых не доминирует, не вносит в сумму определяющего
вклада), имеет распределение, близкое к нормальному.

\subsection{Приближение Муавра-Лапласа}
$X \sim Bi(n,p)$
\begin{tabular}{c|l}
    Локальное &
    $P(X=k) \approx \frac{1}{\sqrt{2\pi npq}}
    \exp\left(-\frac{(k-np)^{2}}{2npq}\right)$
    \\ \hline
    Глобальное &
    $P(l\leqslant X\leqslant m)
    \approx \Phi_{0}\left(\frac{m-np}{\sqrt{npq}}\right)
    - \Phi_{0}\left(\frac{l-np}{\sqrt{npq}}\right)$
    \\ \hline
    Погрешность &
    $\frac{p^{2}+q^{2}}{\sqrt{npq}}$
    \\ \hline
\end{tabular}


\subsection{Приближение Пуассона}
$X \sim Bi(n,p)$
\begin{tabular}{c|l}
    Локальное &
    $P(X=k)\approx\frac{(np)^{k}e^{-np}}{k!}$
    \\ \hline
    Глобальное &
    $P(l\leqslant X\leqslant m)
    \approx\sum\limits_{k=l}^{m}\frac{(np)^{k}e^{-np}}{k!}$
    \\ \hline
    Погрешность &
    $np^{2}$
    \\ \hline
\end{tabular}

\subsection{Неравенство Чебышева, усиленный вариант}
Пусть $r$-й абсолютный момент СВ $X$ конечен. Тогда выполняется
равенство
\begin{displaymath}
    P(\left|X\right| \geqslant \varepsilon)
    \leqslant M\left[\left|X\right|^{r}\right]/\varepsilon^{r},
    \ \ \varepsilon > 0
\end{displaymath}
\par \textit{Неравенство Чебышева} при $r=2$, $Y=X-m_{X}$
\begin{displaymath}
    P\left\{\left|X-m_{X}\right| \geqslant \varepsilon \right\}
    \leqslant
    \frac{M\left[\left|X-m_{X}\right|^{2}\right]}{\varepsilon^{2}}
    \triangleq \frac{D\left[X\right]}{\varepsilon^{2}}
\end{displaymath}

\section{Статистика}
\subsection{Виды статистик}
\begin{enumerate}
    \itemsep0em
    \item $\overline{X}_{n}=\frac{1}{n}\sum\limits_{k=1}^{n}X_{k}$ -
        \textit{выборочное среднее}
    \item $\overline{S}_{n}^{2}=\frac{1}{n}\sum\limits_{k=1}^{n}(X_{k}
        -\overline{X_{n}})^{2}$ - \textit{выборочная дисперсия}
    \item $\overline{\nu}_{r}(n)=\frac{1}{n}\sum\limits_{k=1}^{n}
        (X_{k})^{r},\ r=1,2,\ldots$ - \textit{выборочный начальный
        момент }$r$-го порядка
    \item $\overline{\mu}_{r}(n)=\frac{1}{n}\sum\limits_{k=1}^{n}
        (X_{k}-\overline{X}_{n})^{r},\ r=1,2,\ldots$ - \textit{выборочный начальный
        момент }$r$-го порядка
    \item $\widehat{F}_{n}(x)=\frac{M_{n}(x)}{n}$ - \textit{выборочная
        (эмпирическая) функция распределения}, где $M_{n}(x)$ -
        случайное число элементов выборки, не превосходящих $x$
    \item $\widehat{k}_{XY}(n)=\frac{1}{n}\sum\limits_{k=1}^{n}
        X_{k}Y_{k}-\overline{X}_{n}\overline{Y}_{n}$ -
        \textit{выборочная ковариация случайных величин } $X$ и $Y$
\end{enumerate}

\subsection{Методы построения точечных оценок параметров}
Мы откуда-то узнали формулу распределения и хотим узнать какие у нее
параметры.
\subsubsection{Метод моментов}
Можно использовать если СВ $X$ имеет конечные начальные моменты
$\nu_{r}$ для всех $r\leqslant m$. Алгоритм:
\begin{enumerate}
    \itemsep0em
    \item Найти аналитические выражения для начальных моментов
        $\nu_{r}$(по формуле или из данного закона распределения):
        \begin{eqnarray*}
            \nu_{r}(\theta)
            =\M\left\{X^{r}\right\}
            =\int\limits_{-\infty}^{+\infty}x^{r}dF_{X}(x;\theta)
            ,\ \ r=1,\ldots,m
        \end{eqnarray*}
    \item Вычислить соответствующие выборочные начальные моменты
        (вычислять ничего не требуется, только пишешь формулу):
        \begin{eqnarray*}
            \overline{\nu}_{r}(n)
            =\frac{1}{n}\sum\limits_{k=1}^{n}(X_{k})^{r},\ \ r=1,\ldots,m
        \end{eqnarray*}
    \item Составить систему из $m$ уравнения для переменных
        $\{\theta_{1},\ldots,\theta_{m}\}^{T}$, приравнивая
        соответствующие теоретические и выборочные моменты
        \begin{eqnarray*}
            \nu_{r}(\theta)=\overline{\nu}_{r}(n)
            ,\ \ r=1,\ldots,m
        \end{eqnarray*}
    \item Найти решение $\widehat{\theta}_{n}$ системы уравнений -
        \index{Оценка метода моментов}\textit{оценка метода моментов}
\end{enumerate}
\textit{обычно при решении используют первый начальный момент и второй
центральный (мат ожидание и дисперсия соответственно)}

\subsubsection{Метод максимального правдоподобия (ММП)}

\begin{eqnarray*}
    &&L(x_{1},\ldots,x_{n},\theta)=\prod_{k=1}^{n}f_{X}(x_{k},\theta)\\
    &&\overline{L}(x_{1},\ldots,x_{n},\theta)=\ln L(x_{1},\ldots,x_{n},\theta)\\
    &&\frac{\partial \overline{L}}{\partial\theta}=0
\end{eqnarray*}

\subsection{Свойства точечных оценок}
\begin{enumerate}
    \itemsep0em
    \item $\M\left\{\widehat{\theta}_{n}-\theta_{0}\right\}\to 0
        ,\ n \to \infty$ - \textit{асимптотическая несмещенность}
        \subitem Или $\lim\limits_{n\rightarrow \infty}\M
        [\widehat{\theta}_{n}]=\theta$
    \item $\widehat{\theta}_{n} \stackrel{\textmd{P}}{\to }\theta$ -
        \textit{состоятельная}
    \item $\widehat{\theta} \stackrel{\textmd{ п.н. }}{\to }\theta_{0},\
        n \to \infty$ - \textit{сильная состоятельность}
    \item $\widehat{\theta}_{n} \stackrel{\textmd{с.к.}}{\to }\theta$
        ($\lim\limits_{n\rightarrow
        \infty}M[(\widehat{\theta}_{n}-\theta)^{2}]=0$) -
        \textit{состоятельная в среднем квадратическом}
    \item $M[(\widehat{\theta}_{n}-\theta)^{2}] \leqslant
        M[(\widetilde{\theta}_{n}-\theta)^{2}]
        \ \ \forall\widetilde{\theta}_{n}$ - \textit{оптимальная в
        среднем квадратическом}
    \item $\exists \textmd{ детерменированная последовательность }
        C_{n}$: $C_{n}\left(\widehat{\theta}_{n}-\theta\right)
        \stackrel{d}{\to }u\sim N(0,1)$ - \textit{асимптотически
        нормальная}
    \item $\sqrt{ni(\theta_{0})}(\widehat{\theta}_{n}-\theta_{0})
        \stackrel{\textmd{d}}{\to }\xi \sim N(0,1),\ n \to \infty$ -
        \textit{асимптотическая нормальность}
\end{enumerate}

\subsection{Доказательство свойств точечных оценок}

\begin{itemize}
    \itemsep0em
    \item Асимптотическая несмещенность:
        \subitem Для доказательства $\lim\limits_{n\rightarrow \infty}\M
        [\widehat{\theta}_{n}]=\theta$, надо найти закон распределения
        $\widehat{\theta}_{n}$
        \subitem Закон распределения $\widehat{\theta}_{n}$ найдем из
        функции распредления
        \subitem Функция распределения пытаемся вывести через
        определение:
        \begin{eqnarray*}
            F_{\widehat{\theta}_{n}}(x)=P(\widehat{\theta}_{n}\leqslant x)
        \end{eqnarray*}
    \item Сильная состоятельность, состоятельность: Используем теорему Колмогорова
    \item С.к. состоятельность:
        \subitem По определению имеем уравнение
        \begin{eqnarray*}
            \M [(\widehat{\theta}_{n}-\theta)^{2}] \to 0,\ n\to\infty
        \end{eqnarray*}
        \subitem Раскрываем скобки и получаем
        \begin{eqnarray*}
            \M [\widehat{\theta}_{n}^{2}]-2\M
            [\widehat{\theta}_{n}]\theta +\theta^{2}
        \end{eqnarray*}
\end{itemize}

\subsection{Эффективность точечной оценки}

\par Распределение \textit{регулярно}, если: (непрерывный случай)
\begin{enumerate}
    \itemsep0em
    \item $\sqrt{p(x,\theta)}$ непрерывно дифференцироема по $\theta\in
        \textmd{\textcircled{H}}$ для почти всех $x$ по мере Лебега
    \item $i(\theta)=\int\limits_{-\infty}^{+\infty}
        \left(\frac{\partial\ln
        p(x,\theta)}{\partial\theta}\right)^{2}p(x,\theta)dx$ - конечно,
        положительно, непрерывно по $\theta\in \textmd{\textcircled{H}}$
\end{enumerate}

\par Распределение \textit{регулярно}, если: (непрерывный случай)
\begin{enumerate}
    \itemsep0em
    \item $\sqrt{p(x,\theta)}$ непрерывно дифференцироема по $\theta\in
        \textmd{\textcircled{H}},\ \forall x\in\chi$
    \item $i(\theta)=\sum\limits_{i=1}^{m}
        \left(\frac{\partial\ln
        P(X=x_{i},\theta)}{\partial\theta}\right)^{2}
        p(X=x_{i},\theta)$
\end{enumerate}

Несмещённая оценка $\widehat{\theta}_{n}$ назыв\textbf{ается эффективной},
если
\begin{eqnarray*}
    \M \left[\left(\widehat{\theta}_{n}-\theta\right)^{2}\right]
    =\frac{1}{ni(\theta)}
\end{eqnarray*}
(должны быть выполнены условия регулярности)


\end{document}
