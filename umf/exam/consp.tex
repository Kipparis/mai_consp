\documentclass{article}[12pt]


%Russian-specific packages
%--------------------------------------
\usepackage[T2A]{fontenc}
\usepackage[utf8]{inputenc}
\usepackage[russian]{babel}
%--------------------------------------

%Hyphenation rules
%--------------------------------------
\usepackage{hyphenat}
\hyphenation{ма-те-ма-ти-ка вос-ста-нав-ли-вать}
%--------------------------------------

\setlength{\parindent}{0em} % space before par
\setlength{\parskip}{1em}   % spacing between pars

\usepackage{enumitem} % configure lists
% \setlist{noitemsep,topsep=0pt} % separation
\setlist{nosep} % separation
\setlist[itemize,1]{label=$\triangleleft$} % triangles looking to the left

% margin from paper borders
\usepackage[margin={0.8in, 0.3in}]{geometry}

\usepackage{amsmath}
\usepackage{amssymb}
\usepackage{amsfonts}
% 1 - Name
% 2 - Goes to the index
% 3 - Definition
% internal - some text about teorem
% \newenvironment{theorem}[3]{
% beginning
%     \vspace{0.5\parindent}
%     \par\index{#2}\emph{\underline{Th.} \textbf{#1} #3.}  % teorem's name
%     \par
% }{
% ending
    % \vspace{0.2\parindent}
% }

% 1 - Name
% 2 - Goes to index
% internal - Name's definition
\newenvironment{mydef}[2]{
% beginning
    \vspace{0.5\parindent}
    \par \index{#2} \emph{\underline{Def.} \textbf{#1}} -
}{
% ending
    \vspace{0.2\parindent}
}

% TODO
% \newcommand{hidefinition}

% 1 - Name
% 2 - Given
% 3 - Task
% internal - solution
\newenvironment{example}[3]{
    \vspace{0.5\parindent}
    \par\emph{\underline{Ex.}} #1
    \par \textbf{Given:} #2
    \par \textbf{Find:} #3
    \par \textbf{Solution:}
}{
    \vspace{0.5\parindent}
}

% displaystyle в каждом мас моде
\everymath{\displaystyle}

% TODO: command for drawing box around text (so I can see why some artifact are

% TODO:
% \newcommand{grad}

% indent first line after section
\usepackage{indentfirst}

\usepackage{makeidx}
\makeindex

% keep this in the very bottom
% \usepackage[pdftex]{hyperref}
% remove red boxes around links
\usepackage[pdftex,pdfborderstyle={/S/U/W 0}]{hyperref} % this disables the boxes around links



\begin{document}
\tableofcontents
\newpage
\printindex
\newpage

\section{Основные определения в УМФ}
\begin{definition}
    {Уравнение в частных производных}
    {Уравнение в частных производных}
    уравнение, связывающее независимые переменные $x,y,z,t$, независимую
    функцию $u(x,y,z,t)$ и ее частные производные по независимым
    переменным:
    \begin{eqnarray*}
        F(x,y,z,t,u,u_{x},u_{y},u_{z},u_{xx},u_{xy},u_{xz},u_{yy},u_{yz},
        u_{yt}, u_{yy}, u_{tt})=0
    \end{eqnarray*}
\end{definition}

\begin{definition}
    {Порядок УЧП}
    {Порядок УЧП}
    наибольший порядок частной производной, входящей в это уравнение
\end{definition}

\begin{definition}
    {Отличия решений УМФ от решений ОДУ}
    {Отличия решений УМФ от решений ОДУ}
    Общее решение ОДУ содержит \textit{произвольные постоянные}, общее
    решение уравнения УМФ содержит произвольные функции.
\end{definition}

\begin{definition}
    {Линейное УЧП}
    {Линейное УЧП}
    неизвестная функция и все ее частные производные входят линейным
    образом, то есть уравнение образовано суммой слагаемых, в каждое из
    которых искомая функция и ее производные входят в первой степени и в
    которых (слагаемых) отсутствуют произведения искомой функции и ее
    производных или произведение производных
\end{definition}

\begin{definition}
    {Квазилинейное УЧП}
    {Квазилинейное УЧП}
    ур-е линейное относительно старших производных
\end{definition}

\begin{definition}
    {Однородное УЧП}
    {Однородное УЧП}
    правая часть ($f(x,y,z,t)$) тождественно равна нулю.
    \textit{Неоднородная} в противном случае
\end{definition}

\begin{definition}
    {Общее решение}
    {Общее решение}
    хз
\end{definition}

\begin{definition}
    {Общий интеграл}
    {Общий интеграл}
    хз
\end{definition}

\section{Характеристическое уравнение для линейного УМФ второго
    порядка, характеристики, классификация УМФ}

\section{Приведение УМФ 2-го порядка к каноническому виду, условие
приведения}
\subsection{Приведение УМФ 2-го порядка к каноническому виду}
Задача: найти новые переменные $\xi$ и $\eta$, связанные со старыми $x$
и $y$ соотношениями
\begin{eqnarray*}
    \xi=\varphi(x,y),\ \ \eta=\psi(x,y)
\end{eqnarray*}
Далее сделаем замену в уравнении (для этого порадобиться вычислить новые
частные производные)
\begin{eqnarray*}
    u_{x}=u_{\xi}\xi_{x}+u_{\eta}\eta_{x}
    ,\ \ u_{y}=u_{\xi}\xi_{y}+u_{\eta}\eta_{y}\\
    \ldots
\end{eqnarray*}


\subsection{Условие приведения}
Функции перехода должны быть непрерывны вместе со своими частными
производными до второго порядка включительно по переменным $x$ и $y$. Якобиан
должен быть отличен от нуля (обеспечивает существование обратного преобразования)

\section{Уравнение теплопроводности}
\textbf{Уравнение теплопроводности}:
\begin{eqnarray*}
    \frac{\partial}{\partial x}\left(k\frac{\partial u}{\partial x}\right)
    + F(x,t)=c\rho\frac{\partial u}{\partial t}
\end{eqnarray*}
\begin{enumerate}
    \itemsep0em
    \item Закон Фурье. Если температура тела неравномерна, то в нем
        возникают тепловые потоки, направленные из мест с более высокой
        температурой в места с более низкой температурой;
    \item Внутри стержня может возникать или поглощаться тепло;
    \item Количество тепла, которое необходимо сообщить однородному
        телу, чтобы повысить его температуру на $\Delta u$.
\end{enumerate}

Неокторые частные случаи:
\begin{enumerate}
    \itemsep0em
    \item Если \textit{стержень однороден}, то $k,c,\rho = const$, поэтому
        уравнение записывают в виде
        \begin{eqnarray*}
            u_{t}=a^{2}u_{xx}+f(x,t),
        \end{eqnarray*}
        где $a^{2}=\frac{k}{c\rho}$, $f(x,t)=\frac{F(x,t)}{c\rho}$
    \item Плотность тепловых источников может записеть от температуры
        окружающей среды:
        \begin{eqnarray*}
            F=F_{1}(x,t)-h(u-\theta)
        \end{eqnarray*}
        Если \textit{стержень однороден}, то уравнение теплопроводности с \textit{боковым
        теплообменом} имеет следующий вид:
        \begin{eqnarray*}
            u_{t}=a^{2}u_{xx}-au+f(x,t)
        \end{eqnarray*}
        где $a=\frac{h}{c\rho}$,
        $f(x,t)=a\theta(x,t)+\frac{F_{1}(x,t)}{c\rho}$
    \item Теплообномена с окружающей средой нет, то ур-е будет в виде:
        \begin{eqnarray*}
            u_{t}=a^{2}u_{xx}
        \end{eqnarray*}
\end{enumerate}

\section{Задача Коши о распределении температуры на бесконечной прямой}
\textit{Формулировка:} найти решение уравнения теплопроводности в области
$-\infty < x < \infty$ и $t \geqslant t_{0}$, удовлетворяющее условию
\begin{eqnarray*}
    u(x,t_{0})=\varphi(x)\ \ (-\infty < x < +\infty)
\end{eqnarray*}
где $\varphi(x)$ - заданная функция

\section{Первая краевая задача для полубесконечного стержня}
\textit{Формулировка:} найти решение уравнения теплопроводности в области $0 < x
< \infty$ и $t \geqslant t_{0}$, удовлетворяющее условиям
\begin{eqnarray*}
    u(x,t_{0}) & = & \varphi(x)\ \ (0 < x < \infty)\\
    u(0,t) & = & \mu(t)\ \ (t \geqslant t_{0})
\end{eqnarray*}
где $\varphi(x)$ и $\mu(t)$ - заданные функции

\section{Краевые задачи без начальных условий}
Формулировка: найти решение теплопроводности для $0 \leqslant x
\leqslant l$ и $-\infty < t$, удовлетворяющее условиям ($\mu$ - заданная
функция):
\begin{eqnarray*}
    u(0,t) & = & \mu_{1}(t),\\
    u(l,t) &=&\mu_{2}(t).
\end{eqnarray*}
для полубесконечного стержня
\begin{eqnarray*}
    u(0,t)=\mu(t)
\end{eqnarray*}


\section{Первая краевая задача для ограниченной области}
Решение первой краевой задачи - ф-ция $u(x,t)$, обладающая св-вами:
\begin{enumerate}
    \itemsep0em
    \item $u(x,t)$ - определена и непрерывна в замкнутой области
        \begin{eqnarray*}
            0\leqslant x\leqslant l,\ \ t_{0}\leqslant t\leqslant T;
        \end{eqnarray*}
    \item $u(x,t)$ удовлетворяет уравнению теплопроводности в открытой
        области:
        \begin{eqnarray*}
            0<x<l,\ \ t_{0}<t<T;
        \end{eqnarray*}
    \item $u(x,t)$ удовлетворяет начальному и граничным условиям, т.е.:
        \begin{eqnarray*}
            u(x,t_{0})=\varphi(x),
            \ \ u(0,t)=\mu_{1}(t),
            \ \ u(l,t)=\mu_{2}(t)
        \end{eqnarray*}
        где $\varphi(x)$, $\mu_{1}(t)$, $\mu_{2}(t)$ - непрерывные
        ф-ции, удовлетворяющие условиям сопряжения:
        \begin{eqnarray*}
            \varphi(0)=\mu_{1}(t_{0})\ \ \left[=u(0,t_{0})\right]
            \ \ \textmd{и}
            \ \ \varphi(l)=\mu_{2}(t_{0})\ \ \left[=u(l,t_{0})\right]
        \end{eqnarray*}
\end{enumerate}

\section{Принцип максимального значения}
Будем рассматривать уравнение с постоянными коэффициентами
\begin{eqnarray*}
    v_{t}=a^{2}v_{xx}+\beta v_{x}+\gamma v
\end{eqnarray*}
которое подстановкой
\begin{eqnarray*}
    v=e^{\mu x + \lambda t}\cdot u\ \ \textmd{при}
    \ \ \mu=-\frac{\beta}{2a^{2}},
    \ \ \lambda=\gamma-\frac{\beta^{2}}{4a^{2}}
\end{eqnarray*}
приводится к виду
\begin{eqnarray*}
    u_{t}=a^{2}u_{xx}
\end{eqnarray*}

\index{Теорема о принципе максимального значения}
\textit{Формулировка:} Если функция $u(x,t)$, определенная и непрерывная
в замкнутой области $0\leqslant t \leqslant T$ и $0\leqslant x \leqslant
l$, удовлетворяет уравнению теплопроводности
\begin{eqnarray*}
    u_{t}=a^{2}u_{xx}
\end{eqnarray*}
в точках области $0 < x < l$, $0 < t \leqslant T$, то максимальное и
минимальное значения функции $u(x,t)$ достигаются или в начальный
момент, или в точке границы $x=0$ либо $x=l$
\par Физический смысл: если температура на границе и в начальный момент
не превосходит некоторого значения $M$, то при отсутствии источников
внутри тела не может создаваться температура, большая $M$

\section{Теорема единственности (следствие из принципа максимального
значения)}
\index{Следствия из принципа максимального значения}
Формулировка: если две функции $u_{1}(x,t)$ и $u_{2}(x,t)$, определенные
и непрерывные в области $0\leqslant x\leqslant l$, $0\leqslant
t\leqslant T$, удовлетворяют уравнению теплопроводности:
\begin{eqnarray*}
    u_{t}=a^{2}u_{xx}+f(x,t)\ \ (\textmd{для}\ 0<x<l,\ t> 0)
\end{eqnarray*}
одинаковым начальным и краничным условиям:
\begin{eqnarray*}
    u_{1}(x,0)=u_{2}(x,0)=\varphi(x),
    u_{1}(0,t)=u_{2}(0,t)=\varphi(x),
    u_{1}(l,t)=u_{2}(l,t)=\varphi(x),
\end{eqnarray*}
то $u_{1}(x,t)\equiv u_{2}(x,t)$
\begin{center}
    Доказательство единственности решения (\textit{стр 206 Тихонов})
\end{center}
Пусть функция $v(x,t)$ - разность $u_{1}$ и $u_{2}$. Отсюда знаем что
$v(x,t)$ - решение однородного уравнения теплопроводности в этой
области. Значить \textit{принцип максимального значения} применим к этой
функции. Значин максимальное значение достигается либо при $t=0$, либо
при $x=0$, либо при $x=l$, но по условию имеем:
\begin{eqnarray*}
    v(x,0)=0,\ \ v(0,t)=0,\ \ v(l,t)=0
\end{eqnarray*}
отсюда \begin{eqnarray*}
    v(x,t)\equiv 0
\end{eqnarray*}

\begin{enumerate}
    \itemsep0em
    \item Если два решенеия уравнения теплопроводности $u_{1}(x,t)$ и
        $u_{2}(x,t)$ удовлетворяют условиям
        \begin{eqnarray*}
            &u_{1}(x,0)\leqslant u_{2}(x,0),\\
            &u_{1}(0,t)\leqslant u_{2}(0,t),
            \ \ u_{1}(l,t)\leqslant u_{2}(l,t)
        \end{eqnarray*}
        то
        \begin{eqnarray*}
            u_{1}(x,t)\leqslant u_{2}(x,t)
        \end{eqnarray*}
        для всех значений $0\leqslant x \leqslant l$, $0\leqslant
        t\leqslant T$
    \item Если три решения ур-я теплопроводности:
        \begin{eqnarray*}
            u(x,t), \underline{u}(x,t) , \overline{u}(x,t)
        \end{eqnarray*}
        удовлетворяют условиям
        \begin{eqnarray*}
            \underline{u}(x,t)\leqslant u(x,t)\leqslant
            \overline{u}(x,t)
            \ \ \textmd{при}\ \ t=0,\ \ x=0,\ \ x=l
        \end{eqnarray*}
        то эти же неравенства выполняются тождественно, т.е. для всех
        $x$, $t$ из области $0\leqslant x\leqslant l$, $0\leqslant
        t\leqslant T$
    \item Если для двух решений ур-я теплопроводности $u_{1}(x,t)$ и
        $u_{2}(x,t)$ имеет место нер-во
        \begin{eqnarray*}
            \left|u_{1}(x,t)-u_{2}(x,t)\right|\leqslant \varepsilon
            \ \ \textmd{для}\ \ t=0,\ \ x=0,\ \ x=l
        \end{eqnarray*}
        то оно тождественно, т.е. имеет место для всех $x$, $t$ из
        области $0\leqslant x\leqslant l$, $0\leqslant t\leqslant T$
\end{enumerate}

\section{Теорема единственности для бесконечной прямой}
\textit{Формулировка:} Если $u_{1}(x,t)$ и $u_{2}(x,t)$ - непрерывные,
ограниченные во всей области изменения переменных $(x,t)$ функции,
удовлетворяющие уравнению теплопроводности:
\begin{eqnarray*}
    u_{2}=a^{2}u_{xx}\ \ (-\infty < x < +\infty,\ t\geqslant 0)
\end{eqnarray*}
и условию
\begin{eqnarray*}
    u_{1}(x,0)=u_{2}(x,0)\ \ (-\infty<x<+\infty)
\end{eqnarray*}
то
\begin{eqnarray*}
    u_{1}(x,t)\equiv u_{2}(x,t)\ \ (-\infty<x<+\infty,\ t\geqslant 0)
\end{eqnarray*}

\section{Задача Штурма-Лиувилля}

\section{Метод разделения переменных}
\subsection{Однородная краевая задача}
\begin{eqnarray*}
    u_{t}=a^{2}u_{xx}+f(x,t)
\end{eqnarray*}
с начальным условием
\begin{eqnarray*}
    u(x,0)=\varphi(x)
\end{eqnarray*}
и граничными условиями
\[
\left.
    \begin{array}{c}
        u(0,t)=\mu_{1}(t),\\
        u(l,t)=\mu_{2}(t)
    \end{array}
\right\}\ \ (t\geqslant 0)
\]
\begin{center}
    \textbf{\textit{Решаем вспомогательную задачу}}
\end{center}

Для начала найдем непрерывное в замкнутой области ($0\leqslant
x\leqslant l$, $0\leqslant t\leqslant T$) решение однородного уравнения
\begin{eqnarray*}
    u_{t}=a^{2}u_{xx},\ \ 0<x<l,\ \ 0<t\leqslant T
\end{eqnarray*}
удовлетворяющее начальному условию
\begin{eqnarray*}
    u(x,0)=\varphi(x),\ \ 0\leqslant x\leqslant l
\end{eqnarray*}
и однородным граничным условиям
\begin{eqnarray*}
    u(0,t)=0,\ \ u(l,t)=0\ \ 0\leqslant t\leqslant T
\end{eqnarray*}
это решение не дождественно равно нулю, и представимо в виде:
\begin{eqnarray*}
    u(x,t)=X(x)T(t)
\end{eqnarray*}



\end{document}
