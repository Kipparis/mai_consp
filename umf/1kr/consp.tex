\documentclass{article}[12pt]


%Russian-specific packages
%--------------------------------------
\usepackage[T2A]{fontenc}
\usepackage[utf8]{inputenc}
\usepackage[russian]{babel}
%--------------------------------------

%Hyphenation rules
%--------------------------------------
\usepackage{hyphenat}
\hyphenation{ма-те-ма-ти-ка вос-ста-нав-ли-вать}
%--------------------------------------

\setlength{\parindent}{0em} % space before par
\setlength{\parskip}{1em}   % spacing between pars

\usepackage{enumitem} % configure lists
% \setlist{noitemsep,topsep=0pt} % separation
\setlist{nosep} % separation
\setlist[itemize,1]{label=$\triangleleft$} % triangles looking to the left

% margin from paper borders
\usepackage[margin={0.8in, 0.3in}]{geometry}

\usepackage{amsmath}
\usepackage{amssymb}
\usepackage{amsfonts}
% 1 - Name
% 2 - Goes to the index
% 3 - Definition
% internal - some text about teorem
% \newenvironment{theorem}[3]{
% beginning
%     \vspace{0.5\parindent}
%     \par\index{#2}\emph{\underline{Th.} \textbf{#1} #3.}  % teorem's name
%     \par
% }{
% ending
    % \vspace{0.2\parindent}
% }

% 1 - Name
% 2 - Goes to index
% internal - Name's definition
\newenvironment{mydef}[2]{
% beginning
    \vspace{0.5\parindent}
    \par \index{#2} \emph{\underline{Def.} \textbf{#1}} -
}{
% ending
    \vspace{0.2\parindent}
}

% TODO
% \newcommand{hidefinition}

% 1 - Name
% 2 - Given
% 3 - Task
% internal - solution
\newenvironment{example}[3]{
    \vspace{0.5\parindent}
    \par\emph{\underline{Ex.}} #1
    \par \textbf{Given:} #2
    \par \textbf{Find:} #3
    \par \textbf{Solution:}
}{
    \vspace{0.5\parindent}
}

% displaystyle в каждом мас моде
\everymath{\displaystyle}

% TODO: command for drawing box around text (so I can see why some artifact are

% TODO:
% \newcommand{grad}

% indent first line after section
\usepackage{indentfirst}

\usepackage{makeidx}
\makeindex

% keep this in the very bottom
% \usepackage[pdftex]{hyperref}
% remove red boxes around links
\usepackage[pdftex,pdfborderstyle={/S/U/W 0}]{hyperref} % this disables the boxes around links



\begin{document}
\section{Классификация}
\begin{definition}
    {Уравнение с частными производными 2-го порядка с двумя независимыми
    переменными $x$, $y$}
    {Уравнение с частными производными 2-го порядка с двумя независимыми
    переменными $x$, $y$}
    соотношение между неизвестной функцией $u(x,y)$ и её частными
    производными до 2-го порядка включительно:
    \begin{displaymath}
        F(x,y,u,u_{x},u_{y},u_{xx},u_{xy},u_{yy})=0
    \end{displaymath}
\end{definition}
\begin{definition}
    {Линейное уравнение относительно старших производных}
    {Линейное уравнение относительно старших производных}
    уравнение, имеющее вид
    \begin{displaymath}
        a_{11}u_{xx}+2a_{12}u_{xy}+a_{22}u_{yy}
        +F(x,y,u,u_{x},u_{y})=0
    \end{displaymath}
    где $a_{11}$, $a_{12}$, $a_{22}$ - функции от $x$, $y$.
\end{definition}
\begin{definition}
    {Квазилинейное уравнение}
    {Квазилинейное уравнение}
    уравнение, имеющее вид
    \begin{displaymath}
        a_{11}u_{xx}+2a_{12}u_{xy}+a_{22}u_{yy}
        +F(x,y,u,u_{x},u_{y})=0
    \end{displaymath}
    где $a_{11}$, $a_{12}$, $a_{22}$ - функции от $x$, $y$, $u$,
    $u_{x}$, $u_{y}$.
\end{definition}
\begin{definition}
    {Линейное уравнение}
    {Линейное уравнение}
    уравнение, линейное как относительно старших производных $u_{xx}$,
    $u_{xy}$, $u_{yy}$, так и относительно функции $u$ и ее первых
    производных $u_{x}$, $u_{y}$:
    \begin{displaymath}
        a_{11}u_{xx}+2a_{12}u_{xy}+a_{22}u_{yy}
        +b_{1}u_{x}+b_{2}u_{y}+cu+f=0
    \end{displaymath}
    где $a_{11}$, $a_{12}$, $a_{22}$, $b_{1}$, $b_{2}$, $c$, $f$ -
    функции только $x$ и $y$.
    \par Если коэффициекнты не зависят от $x$ и
    $y$, то оно представляет собой \textit{линейное уравнение с
    постоянными коэффициентами}\index{Линейное уравнение с постоянными
    коэффициентами}.
    \par Если $f(x,y)=0$, то уравнение
    \textit{однородное}\index{Однородное уравнение}
\end{definition}
\section{В частных производных второго порядка}
Уравнение \textbf{\textit{линейное относительно старших производных}} можно привести к
канноническому виду с помощью замены переменных на
$\xi$ и $\eta$ ($\xi=\varphi(x,y)$, $\eta=\psi(x,y)$)
\par Изначальное уравнение имеет вид
\begin{displaymath}
    a_{11}u_{xx}+2a_{12}u_{xy}+a_{22}u_{yy}+F(x,y,u,u_{x},u_{y})=0
\end{displaymath}
\par Составляем \textit{характеристическую функцию}:
\begin{displaymath}
    a_{11}(dy)^{2}-2a_{12}dxdy+a_{22}(dx)^{2}=0
\end{displaymath}
\par Находим области гиперболичности, параболичности, элиптичности
функции.
\begin{displaymath}
    D=a_{12}^{2}-a_{11}a_{22}\left\{
        \begin{array}{ll}
            > 0,&\textmd{гиперболичноский тип}\\
            = 0,&\textmd{параболичноский тип}\\
            < 0,&\textmd{элиптичноский тип}
        \end{array}
    \right.
\end{displaymath}
\par Для каждой области интегрируем характеристическую функцию. Общие
интегралы (решение кроме постоянной $C$) и представляют функцию для замены.
\begin{displaymath}
    \xi=\varphi(x,y),\ \ \ \eta=\psi(x,y)
\end{displaymath}
\textit{например, если в решении получилось} $y=\pm x + C$ \textit{тогда}
$\xi=y+x,\ \eta=y-x$
\parвозможно к более хорошему виду приведет замена
\begin{displaymath}
    \alpha=\frac{\xi-\eta}{2},
    \ \ \beta=\frac{\xi+\eta}{2}\textmd{ или }\frac{\xi+\eta}{2i}
\end{displaymath}
можно модифицировать преобразования для более хорошего результата,
главное - выполнение условия:
\begin{displaymath}
    % \arraycolsep=1.4pt\def\arraystretch{2.2}
    \left|\begin{array}{cc}
        \varphi_{x} & \psi_{x} \\
        \varphi_{y} & \psi_{y}
    \end{array}\right| \neq 0
\end{displaymath}
\par После нахождения замены переменных, вычисляем частные производные
до второго порядка
\begin{eqnarray*}
    u_{x}=u_{\xi}\xi_{x}+u_{\eta}\eta_{x}\\
    u_{y}=u_{\xi}\xi_{y}+u_{\eta}\eta_{y}\\
    \ldots
\end{eqnarray*}
\par \textit{Коэффициенты} $\xi_{x},\xi_{y}$ \textit{- то, что мы
считаем и подставляем. Коэффициенты} $u_{\xi},u_{\eta}$ \textit{- то,
что мы не трогаем}
\par Подставляем частные производные в исходное уравнение и должен
получится что-то из каннонических форм:
\begin{center}
\begin{tabular}{ |l|c| }
    Тип & Каннонический вид \\
    \hline
    Гиперболический & $u_{\xi\eta}=\Phi(\xi,\eta,u,u_{\xi},u_{\eta})$ \\
                    & $u_{\alpha\alpha}-u_{\beta\beta}=\Phi$\\
    \hline
    Параболический & $u_{\eta\eta}=\Phi(\xi,\eta,u,u_{\xi},u_{\eta})$\\
                   & $u_{\xi\xi}=\Phi(\xi,\eta,u,u_{\xi},u_{\eta})$\\
    \hline
    Эллиптический & $u_{\alpha\alpha}+u_{\beta\beta}
                        =\Phi(\xi,\eta,u,u_{\xi},u_{\eta})$ \\
    \hline
\end{tabular}
\end{center}

Если каноническая форма не выходит, значит при решении допущена
ошибка.

\section{Линейные с постоянными коэффициентами}
Решаем обычное уравнение. В приведенном уравнении раскрываем все скобки и
переносим всё по одну сторону от равно.
\par Затем производим замену
\begin{eqnarray*}
    &&u(\xi,\eta)=V(\xi,\eta)e^{\lambda\xi+\mu\eta}
    \\&&u_{\xi}=(\lambda V+V_{\xi})e^{\lambda\xi+\mu\eta}
    \\&&u_{\eta}=(\mu V+V_{\eta})e^{\lambda\xi+\mu\eta}
    \\&&u_{\xi\xi}=(\lambda^{2}V+2\lambda
    V_{\xi}+V_{\xi\xi})e^{\lambda\xi+\mu\eta}
    \\&&u_{\xi\eta}=(\lambda\mu V+\mu V_{\xi}+\lambda
    V_{\eta}+V_{\xi\eta})e^{\lambda\xi+\mu\eta}
    \\&&u_{\eta\eta}=(\mu^{2}V+2\mu
    V_{\eta}+V_{\eta\eta})e^{\lambda\xi+\mu\eta}
\end{eqnarray*}
выносим экспоненту за скобку и группируем все относительно относительно
$V$ и его частных производных.
\par У нас образуются скобочки в которые входят коэффициенты
$\mu,\lambda$, которые мы приравниваем к нулю. Отсюда и находим
коэффициенты $\mu,\lambda$.
\par Получаем решение системы. Например:
\begin{displaymath}
    \left\{\begin{array}{ll}
            V_{\eta\eta}-\frac{1}{9}V_{\xi}=0
            \\u=V(\xi,\eta)e^{-\frac{1}{9}\xi -\frac{1}{9}\eta},
            & \xi=-\frac{1}{9},\ \eta=-\frac{1}{9}
    \end{array}\right.
\end{displaymath}


\end{document}
