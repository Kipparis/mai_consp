\documentclass{article}[12pt]

\setlength{\parindent}{0em} % space before par
\setlength{\parskip}{1em}   % spacing between pars

\usepackage{enumitem} % configure lists

\raggedbottom


%Russian-specific packages
%--------------------------------------
\usepackage[T2A]{fontenc}
\usepackage[utf8]{inputenc}
\usepackage[russian]{babel}
%--------------------------------------

%Hyphenation rules
%--------------------------------------
\usepackage{hyphenat}
\hyphenation{ма-те-ма-ти-ка вос-ста-нав-ли-вать}
%--------------------------------------

\setlength{\parindent}{0em} % space before par
\setlength{\parskip}{1em}   % spacing between pars

\usepackage{enumitem} % configure lists
% \setlist{noitemsep,topsep=0pt} % separation
\setlist{nosep} % separation
\setlist[itemize,1]{label=$\triangleleft$} % triangles looking to the left

% margin from paper borders
\usepackage[margin={0.8in, 0.3in}]{geometry}

\usepackage{amsmath}
\usepackage{amssymb}
\usepackage{amsfonts}
% 1 - Name
% 2 - Goes to the index
% 3 - Definition
% internal - some text about teorem
% \newenvironment{theorem}[3]{
% beginning
%     \vspace{0.5\parindent}
%     \par\index{#2}\emph{\underline{Th.} \textbf{#1} #3.}  % teorem's name
%     \par
% }{
% ending
    % \vspace{0.2\parindent}
% }

% 1 - Name
% 2 - Goes to index
% internal - Name's definition
\newenvironment{mydef}[2]{
% beginning
    \vspace{0.5\parindent}
    \par \index{#2} \emph{\underline{Def.} \textbf{#1}} -
}{
% ending
    \vspace{0.2\parindent}
}

% TODO
% \newcommand{hidefinition}

% 1 - Name
% 2 - Given
% 3 - Task
% internal - solution
\newenvironment{example}[3]{
    \vspace{0.5\parindent}
    \par\emph{\underline{Ex.}} #1
    \par \textbf{Given:} #2
    \par \textbf{Find:} #3
    \par \textbf{Solution:}
}{
    \vspace{0.5\parindent}
}

% displaystyle в каждом мас моде
\everymath{\displaystyle}

% vertical space for mathmode
\setlength{\abovedisplayskip}{0pt}
\setlength{\belowdisplayskip}{0pt}
\setlength{\abovedisplayshortskip}{0pt}
\setlength{\belowdisplayshortskip}{0pt}

% TODO: command for drawing box around text (so I can see why some artifact are

% TODO:
% \newcommand{grad}

% indent first line after section
\usepackage{indentfirst}

\usepackage{makeidx}
\makeindex

% keep this in the very bottom
% \usepackage[pdftex]{hyperref}
% remove red boxes around links
\usepackage[pdftex,pdfborderstyle={/S/U/W 0}]{hyperref} % this disables the boxes around links


\begin{document}
\tableofcontents
\newpage
\printindex
\newpage

\section{Постановка задачи}
\subsection{Вид уравнений теплопроводности}
\paragraph{Теплоизолированная боковая поверхность}
\textit{Если ничего не сказанно, считаем что поверхность
теплоизолированна}
\begin{eqnarray*}
    u_{t}=a^{2}u_{xx}
\end{eqnarray*}
в данном случае уравнение \textit{однородно}. В уравнении
\begin{description}[noitemsep,topsep=0pt]
    \item[$a^{2}=\frac{\lambda}{c\rho}$] - коэффициент температуропроводности
        \subitem $\lambda$ - коэффициент теплопроводности материала
        стержня
        \subitem $c$ - удельня теплоемкость
        \subitem $\rho$ - плотность массы
\end{description}

\paragraph{На боковой поверхности происходит конвективный теплообмен}
\begin{eqnarray*}
    u_{t}=a^{2}u_{xx}+\frac{P\alpha}{Sc\rho}(T_{\textmd{ср}}-T)
    =a^{2}u_{xx}+b(T_{\textmd{ср}}-u)
\end{eqnarray*}
где:
\begin{description}[noitemsep,topsep=0pt]
    \item ${P}$ - периметр сечения
    \item $\alpha$ - к-ф. теплообмена
    \item $S$ - площадь поперечного сечения
    \item $c$ - теплоемкость материала
    \item $\rho$ - плотность материала
    \item $T_{\textmd{ср}}$ - температура среды
    \item $T$ - температура материала
\end{description}
\textit{обычно дробь обозначают одним символом, когда она константа.}

\par Более того, воспользовавшись приведением к более простому виду
(\textit{с помощью замены $u=Ve^{\lambda x+\mu t}$ можно привести
уравнение к однородному относительно $V$})

\subsection{Вид начальных условий}
\paragraph{Начальная температура равна произвольной функции.} \textit{Начальное
условие} \index{Начальное условие}
\begin{eqnarray*}
    u(x,0)=f(x)
\end{eqnarray*}
если $f(x)=0$ - условие \textit{однородно}


\subsection{Вид краевых условий}

\paragraph{Температура концов поддерживается равной произвольной
функции.}
Краевое условие \textit{первого типа}\index{Краевое условие первого типа}
\begin{eqnarray*}
    u(0,t)=\varphi_{1}(t),\ \ u(l,t)=\varphi_{2}(t)
\end{eqnarray*}
если $\varphi_{i}(t)=0$ - данное условие \textit{однородно}

\paragraph{Теплоизолированные концы.} Краевое условие \textit{второго рода}
\index{Краевое условие второго рода}
\begin{eqnarray*}
    u_{x}(0,t)=0,\ \ u_{x}(l,t)=0
\end{eqnarray*}

\paragraph{Тепловой поток.} Краевое условие \textit{второго рода}
\begin{eqnarray*}
    u_{x}=-\frac{q}{\lambda S}
\end{eqnarray*}
$q$ - данный в задаче тепловой поток (то есть надо провести операции
чтобы привести его к нужному виду). Если нагрев идет справа, то
производная должна иметь отрицательный знак. Если нагрев идет слева -
положительный знак. В уравнении:
\begin{description}[noitemsep,topsep=0pt]
    \item $\lambda$ - коэффициент теплопроводности
    \item $S$ - площадь сечения
\end{description}

\section{Методы решений}
\subsection{Однородные краевые условия}
\subsubsection{Через функцию источника}
\paragraph{Однородное уравнение}
\paragraph{Бесконечная прямая, неоднородное начальное условие}
\begin{eqnarray*}
    u(x,t)=\int\limits_{-\infty}^{\infty}G(x,\xi,t)\varphi(\xi)d\xi
\end{eqnarray*}
где
\begin{eqnarray*}
    G(x,\xi,t)
    =\frac{1}{2\sqrt{\pi a^{2}t}}e^{-\frac{(x-\xi)^{2}}{4a^{2}t}}
\end{eqnarray*}

\paragraph{Начальное неоднородное условие}
\begin{eqnarray*}
    u(x,t)=\int\limits_{0}^{l}\varphi(\xi)G(x,\xi,t)d\xi
\end{eqnarray*}
\subparagraph{Однородные краевые условия второго рода}
\begin{eqnarray*}
    G(x,\xi,t)=\frac{1}{l}+\frac{2}{l}\sum\limits_{k=1}^{\infty}
    e^{-a^{2}\frac{\pi^{2}k^{2}}{l^{2}}t}
    \cos{\frac{\pi k}{l}\xi}\cos{\frac{\pi k}{l}x}
\end{eqnarray*}

\paragraph{Однородное начальное условие}

\paragraph{Неоднородное уравнение теплопроводности}
\begin{eqnarray*}
    u(x,t)=\int\limits_{0}^{t}\int\limits_{0}^{l}
    f(\xi,\tau)G(x,\xi,t-\tau)d\xi d\tau
\end{eqnarray*}

\paragraph{И начальное неоднородное, и неоднородное уравнение}
\begin{eqnarray*}
    u(x,t)=\int\limits_{0}^{t}\int\limits_{0}^{l}
    f(\xi,\tau)G(x,\xi,t-\tau)d\xi d\tau
    +\int\limits_{0}^{l}\varphi(\xi)G(x,\xi,t)d\xi
\end{eqnarray*}
\textit{просуммировали два решения: начальное неоднородное без
неоднородности в уравнении и неоднородное в уравнении без начального
неоднородного}



\subsubsection{Метод разделения переменных}
\paragraph{Решаем однородное}
Функция теплопроводности представляется в виде:
\begin{eqnarray*}
    u(x,t)=\sum\limits_{k=1}^{\infty}T_{k}(t)X_{k}(x)
\end{eqnarray*}
или
\begin{eqnarray*}
    u(x,t)=T(t)X(x)
\end{eqnarray*}
Подставляем $u(x,t)=X(x)T(t)$ в уравнение теплопроводности, получаем
\begin{eqnarray*}
    XT'=a^{2}X''T
\end{eqnarray*}
делим на $XTa^{2}$, получаем
\begin{eqnarray*}
    \frac{X''}{X}=\frac{T'}{a^{2}T}=\lambda^{2}
\end{eqnarray*}
отсюда получаем два уравнения:
\begin{eqnarray*}
    X''-\lambda^{2}X=0,\ \ T'+a^{2}\lambda^{2}T=0
\end{eqnarray*}

\paragraph{Находим собственные значения, собственные функции}
Рассматриваем несколько знаков при $\lambda^{2}$: $-\lambda^{2}$,
$+\lambda^{2}$ и $0$. Для каждого из них решаем получившееся ДУ.
Выбираем такие константы, чтобы выполнялись краевые условия, а сама
функция $X(x)$ не равнялась нулю. Там же будут периодические функции,
которые и будут являтся собственными функциями.
\par Собственные значения будут $\lambda_{k}$, собственными функциями будут
$X_{k}$ (\textit{находим исходя из краевых условий})

\paragraph{Находим значения $T$ или $T_{k}$}
Из второго выражения получаем значения для $T$
\begin{eqnarray*}
    T'+\lambda^{2}a^{2}T=0 \Rightarrow
    T_{k}=C_{k}\cdot e^{-a^{2}\lambda_{k}^{2}t}
\end{eqnarray*}

\paragraph{Подставляем в уравнение теплопроводности}
Выразив $X_{k}(x)$ и $T_{k}(t)$ подставляем их в уравнение
$u(x,t)=\sum\limits_{k=0}^{\infty}T_{k}(t)X_{k}(x)$

\paragraph{Удовлетворяем начальному условию}
Это может быть либо \textit{однородное} начальное условие: $u(x,0)=0$, в этом
случае получится несложное ДУ. Либо \textit{неоднородное} начальное условие:
$u(x,0)=\varphi(x)$
\begin{eqnarray*}
    u(x,0)=\varphi(x)=\sum\limits_{k=0}^{\infty}T_{k}(0)X_{k}(x)
    =\sum\limits_{k=0}^{\infty}\varphi_{k}X_{k}(x)
\end{eqnarray*}
где последняя сумма - разложение $\varphi(x)$ по собственной функции
(синусу либо косинусу)
\par \textit{Считаем коэффициенты}
\begin{eqnarray*}
    C_{k}=\varphi_{k}=\frac{1}{\Vert X_{k} \Vert^{2}}\int\limits_{0}^{l}
    \varphi(\xi)X_{k}(\xi)d\xi
\end{eqnarray*}
Подставляем эти коэффициенты в функцию теплоты:
\begin{eqnarray*}
    u(x,t)=\sum\limits_{k=0}^{\infty}T_{k}(t)X_{k}(x)
\end{eqnarray*}


\paragraph{Решаем неоднородное}
Так же можно использовать $u=V\cdot e^{\lambda x+\mu t}$
\par Функцию $f(x,t)$ раскладываем в ряд фурье по синусам или косинусам (тому
же, что и собственные функции у $u(x,t)$)
\par Подставляем полученный ряд в однородную формулу теплопроводности
(например, если уравнение имеет вид $u_{t}=a^{2}u_{xx}+f(x,t)$, то мы
считаем первую производную по $t$ и вторую производную по $x$ и
подставляем полученные ряды в $u_{t}=a^{2}u_{xx}$)
\par Получаем неоднородное дифференциальное уравнение относительно
$T_{k}$. Сначала ищем однородное, потом частное решения и общее решение
будет являтся суммой. (Выражаем $C_{k}$ из однородного, подставляем его
в частное, суммируем оба решения). Затем используем краевое условия для
определения коэффициентов $C_{k}$
\par Записываем общее решение (есть $T_{k}$ и есть $X_{k}$).
\textit{Можно поменять местами порядок интегрирования и суммирования и
получим формулу через функцию источника}
\paragraph{Последний штрих}
Находим коэффициенты $C_{k}$ используя начальные условия.
(\textit{Может понадобится разложить какую-нибудь функцию в ряд по
косинусам/синусам, если константы образуют сумму})
\paragraph{Пример разложения ряда для начального неоднородного краевого условия}
\begin{eqnarray*}
    &&u(x,0)=C_{0}+\sum\limits_{k=1}^{\infty}C_{k}\cdot\cos{\frac{\pi
    k}{l}}x
    =\varphi(x)=\varphi_{0}+\sum\limits_{k=1}^{\infty}\varphi_{k}
    \cdot\cos{\frac{\pi k}{l}}x  \\
    &&C_{0}=\varphi_{0}=\frac{1}{\Vert x_{0}\Vert^{2}}
    \cdot \int\limits_{0}^{l}X_{0}(\xi)\varphi(\xi)d\xi
    =\frac{1}{l}\int\limits_{0}^{l}\varphi(\xi)d\xi\\
    &&C_{k}=\varphi_{k}=\frac{1}{\Vert X_{k}
    \Vert^{2}}\int\limits_{0}^{l}X_{k}(\xi)\varphi(\xi)d\xi
    =\frac{2}{l}\int\limits_{0}^{l}\cos{\frac{\pi
    k}{l}\xi}\varphi(\xi)d\xi
\end{eqnarray*}
Где норма считается как
\begin{eqnarray*}
    &&\Vert X_{0} \Vert^{2}
    =\int\limits_{0}^{l}1d\xi=l\\
    &&\Vert X_{k} \Vert^{2}
    =\int\limits_{0}^{l}\cos^{2}{\frac{\pi k}{l}}\xi d\xi =\frac{l}{2}
\end{eqnarray*}
\textit{В общем случае, если вместо функции стоит собственная функция
или константа, то можно не расскладывать в ряд, а сразу определить
коэффициенты}


\subparagraph{Пример определения коэффициентов}
\begin{eqnarray*}
    C'_{k} & = & f_{k}(t)\cdot e^{g(k)t}\\
    C_{k} & = & \int\limits_{0}^{t}f_{k}(\tau)e^{g(k)\tau}d\tau
\end{eqnarray*}


\subsection{Неоднородное уравнение теплопроводности}
Используем замену $u=Ve^{\lambda x+\mu t}$. Решаем однородное уравнение
теплопроводности

\subsection{Неоднородные краевые условия}
Используем \textit{метод редукции}\index{Метод редукции} Представляем
уравнение теплопроводности в виде:
\begin{eqnarray*}
    u(x,t)=v(x,t)+w(x,t)
\end{eqnarray*}
Тогда постановка задачи будет следующей
\begin{eqnarray*}
    \left\{
    \begin{array}{l}
    v_{t}+w_{t}=a^{2}v_{xx}+a^{2}w_{xx}+f(x,t)\\
    v(x,0)+w(x,0)=\varphi(x)\\
    v(0,t)+w(0,t)=\mu_{1}(t)\\
    v(l,t)+w(l,t)=\mu_{2}(t)\\
    \end{array}
    \right.
\end{eqnarray*}
Разделяем эту систему на две системы, где под функцию $w$ всегда идет
неоднородность краевых условий. Саму ф-цию $w$ выбираем любую,
единственное - должны удовлетворить краевым условиям. Упростим себе
решение, если потребуем, чтобы $w_{xx}=0$ (\textit{возможно \textbf{только для
первых краевых условий} и \textbf{для смешанных краевых условий}}). Получаем
следующее
\begin{eqnarray*}
    &&\left\{
    \begin{array}{l}
        w_{xx}=0\\
        w(0,t)=\mu_{1}(t)\\
        w(l,t)=\mu_{2}(t)
    \end{array}
    \right.\ \ \textmd{далее решаем ДУ чтобы найти }w\\
    &&w(x,t)=C_{1}x+C_{2}\ \ \textmd{находим к.ф. }C_{1},\ C_{2}\\
    &&w(0,t)=C_{2}=\mu_{1}(t)\\
    &&w(l,t)=C_{1}l+\mu_{1}(t)=\mu_{2}(t)\\
    &&C_{1}=\frac{1}{l}\left(\mu_{2}(t)-\mu_{1}(t)\right)
\end{eqnarray*}
Нашли функцию $w$
\begin{eqnarray*}
    w(x,t)=\frac{x}{l}\left(\mu_{2}-\mu_{1}(t)\right)+\mu_{1}(t)
\end{eqnarray*}
если первое краевое - 2-го рода, а второе краевое - первого рода:
\begin{eqnarray*}
    w(x,t)=\mu_{1}(t)(x-l)+\mu_{2}(t)
\end{eqnarray*}

Находим ее значения/производные (для подстановки в систему уравнений,
которую мы разделили)
\begin{eqnarray*}
    &&w_{t}=\frac{x}{l}\left(\mu'_{2}-\mu'_{1}\right)+\mu'_{1}\\
    &&w(x,0)=\frac{x}{l}\left(\mu_{2}(0)-\mu_{1}(0)\right)+\mu_{1}(0)\\
    &&w(x,0)=\frac{x}{l}\left(\mu_{2}(0)-\mu_{1}(0)\right)+\mu_{1}(0)\\
\end{eqnarray*}
В итоге, после подсчета и подстановки $w$ получаем задачу:
\begin{eqnarray*}
&&\left\{
    \begin{array}{l}
        v_{t}=a^{2}+(f(x,t)-w_{t})=a^{2}v_{xx}+\widetilde{f}(x,t)\\
        v(x,0)=\varphi(x)-w(x,0)=\widetilde{\varphi}(x,t)\\
        v(0,t)=0\\
        v(l,t)=0
    \end{array}
\right.
\end{eqnarray*}
Уже знаем как решать \textit{однородные краевые условия, неоднородное
уравнение, неоднородное начальное условие: сумма неоднородного уравнения
и неоднородного начального условия}
\par Напоминаю, решением является сумма $u(x,t)=v(x,t)+w(x,t)$

\subsubsection{Вторые краевые условия}
\begin{eqnarray*}
    &&\left\{
    \begin{array}{l}
        w=C_{1}x^{2}+C_{2}x\\
        w(0,t)=\mu_{1}(t)\\
        w(l,t)=\mu_{2}(t)
    \end{array}
    \right.\ \ \textmd{далее находим }C_{1},\ C_{2}\\
    &&w_{x}(x,t)=2C_{1}x+C_{2}\\
    &&w_{x}(0,t)=C_{2}=\mu_{1}(t)\ \Rightarrow C_{2} = \mu_{1}\\
    &&w_{x}(l,t)=2C_{1}l+\mu_{1}=\mu_{2}\ \Rightarrow
    C_{1}=\frac{1}{2l}(\mu_{2}-\mu_{1})\\
    &&w(x,t)=\frac{x^{2}}{2l}(\mu_{2}(t)-\mu_{1}(t))+\mu_{1}(t)x
    \ \ \textmd{подставили постоянные}\\
    &&w_{xx}=\frac{\mu_{2}-\mu_{1}}{l}
\end{eqnarray*}


\section{Функции источников}
\subsection{Однородные краевые условия}
\paragraph{Для первых краевых условий}
\begin{eqnarray*}
    G(x,\xi,t)
    =\frac{2}{l}\sum\limits_{k=1}^{\infty}
    \sin{\frac{\pi k}{l}x}
    \sin{\frac{\pi k}{l}\xi}
    \cdot e^{-\frac{\pi^{2}k^{2}}{l^{2}}a^{2}t}
\end{eqnarray*}

\paragraph{Для вторых краевых условий}
\begin{eqnarray*}
    G(x,\xi,t)=\frac{1}{l}+\frac{2}{l}\sum\limits_{k=1}^{\infty}
    e^{-\frac{a^{2}\pi^{2}k^{2}}{l^{2}}t}\cos{\frac{\pi k}{l}\xi}
    \cos{\frac{\pi k}{l}x}
\end{eqnarray*}
\paragraph{Для первой и второй}
\begin{eqnarray*}
    G(x,\xi,t)
    =\frac{2}{l}\sum\limits_{k=0}^{\infty}
    e^{-a^{2}(\frac{\pi(2k+1)}{2l})^{2}t}
    \sin{\frac{\pi(2k+1)}{2l}x}
    \cdot\sin{\frac{\pi(2k+1)}{2l}\xi}d\xi
\end{eqnarray*}
\paragraph{Для второй и первой}
\begin{eqnarray*}
    G(x,\xi,t)
    =\frac{2}{l}\sum\limits_{k=0}^{\infty}
    e^{-a^{2}(\frac{\pi(2k+1)}{2l})^{2}t}
    \cos{\frac{\pi(2k+1)}{2l}x}
    \cdot\cos{\frac{\pi(2k+1)}{2l}\xi}d\xi
\end{eqnarray*}

\paragraph{Для третьей краевой задачи}
\subsection{Как выводить}
\begin{eqnarray*}
    G(x,\xi,t)
    =\sum\limits_{k=0}^{\infty}\frac{1}{\Vert X_{k} \Vert^{2}}
    \cdot e^{-a^{2}\lambda_{k}^{2}t}
    \cdot X_{k}(x)\cdot X_{k}(\xi)
\end{eqnarray*}


\section{Собственные функции}
\subsection{Для первых краевых условий (однородных)}
сф: $X_{k}(x)=\sin\frac{\pi k}{l}x$, сз: $\lambda_{k}=\frac{\pi k}{l}$,
где $k=1,\ldots$
\subsection{Для вторых краевых условий (однородных)}
сз: $\lambda_{k}=\frac{\pi k}{l}$, сф: $X_{k}=C_{k}\cos{\frac{\pi k}{l}x}$



\section{Дополнительная инфа}
\subsection{Что может встретиться}
\paragraph{Задача Штурмо-Лиувилля}\index{Задача Штурмо-Лиувилля}
Находим собственные значения и собственные функции
\paragraph{Задачи из учебника}
\begin{description}
    \item 23
    \item 24
    \item 25
\end{description}


\subsection{Решение ДУ}
\paragraph{Примеры решений неоднородного ДУ}
\subparagraph{Однородное решение:}
\begin{eqnarray*}
    T'_{k}+g(k)T_{k}=0\ \Rightarrow T_{\textmd{одн}}=C_{k}\cdot
    e^{-g(k)t}
\end{eqnarray*}
где $g(k)$ - любая функция, $T_{k}$ - функция от $t$

\subparagraph{Частное решение:}
\begin{eqnarray*}
    T'_{k}+g(k)T_{k}= f_{k}(t)\ \Rightarrow T_{\textmd{одн}}=C_{k}(t)\cdot
    e^{-g(k)t}
\end{eqnarray*}
где $g(k)$ - любая функция, $T_{k}$ - функция от $t$, $C_{k}(t)$ -
неопределённая функция от $t$

\subsection{Ряды Фурье}

\paragraph{Пример разложения функции в ряд Фурье}
\begin{eqnarray*}
    f(x,t)=\sum\limits_{k=1}^{\infty}f_{k}(t)\sin\frac{\pi k}{l}x
\end{eqnarray*}
где
\begin{eqnarray*}
    f_{k}(t)=\frac{2}{l}\int\limits_{0}^{l}f(\xi,t)\sin\frac{\pi
    k}{l}\xi d\xi
\end{eqnarray*}

\subsection{Анализ рядов}
\paragraph{Теорема Лейбница}
Остаток ряда не превосходит по абсолютной величине первого из
отброшенных членом (\textit{для знакочередующегося ряда})

\newpage\section{Литература}
\begin{enumerate}
    \item Будак
    \item Тихонов
\end{enumerate}


\end{document}
